\documentclass{article}[11pt]
\usepackage[frenchb,english]{babel}
\usepackage[T1]{fontenc}
\usepackage[utf8]{inputenc}
\usepackage{amsmath,amssymb,latexsym}
\usepackage{times}
\usepackage{float}
\usepackage[left=2cm,right=2cm,top=2cm,bottom=2cm]{geometry}
\frenchbsetup{StandardLists=true} % � inclure si on utilise \usepackage[french]{babel}
\usepackage{enumitem}
\usepackage{fancyhdr}
\usepackage{mathrsfs}
\usepackage{graphicx}
%\usepackage[Algorithme]{algorithm}
%\usepackage{algorithmic}
\usepackage{tikz}
\usepackage{tabularx}
\usetikzlibrary{shapes}
\pagestyle{fancy}
\newcommand{\tr}[1]{{\vphantom{#1}}^{\mathit t}{#1}} 
\renewcommand\headrulewidth{1pt}
\fancyhead[L]{Cours 1�re S}
\fancyhead[R]{Yoann Pietri}
\newcounter{theoremecounter}[subsection]
\usepackage{titlesec}
\setcounter{secnumdepth}{3}% enl�ve la num�rotation apr�s les sections
%\renewcommand\thechapter {\Roman{chapter}}

 \setlength{\parindent}{0pt}

\newcommand{\R}{\mathbb{R}}
\newcommand{\N}{\mathbb{N}}
\newcommand{\Q}{\mathbb{Q}}
\newcommand{\Z}{\mathbb{Z}}
\newcommand{\C}{\mathbb{C}}
\newcommand{\K}{\mathbb{K}}
\newcommand{\eqi}{\Leftrightarrow}
\titleformat{\subsubsection}
   {\normalfont\fontsize{11pt}{13pt}\selectfont\bfseries}% apparence commune au titre et au num�ro
   {\thesubsubsection}% apparence du num�ro
   {1em}% espacement num�ro/texte
   {}% apparence du titre

\tikzstyle{theobox} = [draw=black, very thick,
    rectangle, rounded corners, inner sep=10pt, inner ysep=20pt]
\tikzstyle{theotitle} =[fill=white, text=black,rounded corners,draw=black,very thick]

\fancyhead[L]{Contrôle chapitre 2}


\begin{document}
\center
\Large Contrôle de cours
\flushleft
\center
Fonctions de référence
\flushleft \normalsize
Durée du contrôle : 1h\newline
Ce sujet comporte 2 pages\newline
La calculatrice est autorisée
\subsection*{Exercice 1 (R.O.C., temps conseillé : 10 min) : }
Après avoir rappelé la définition de la fonction inverse et son ensemble de définition, vous montrerez que cette fonction est décroissante sur cet ensemble. Finalement, tracer la courbe représentative de la fonction inverse
\subsection*{Exercice 2 (Etude d'une fonction, temps conseillé : 15-18 min) : }
On étudie la fonction 
$$f:x\mapsto \frac{1}{\sqrt{2x^2-12x+16}}$$
\begin{enumerate}
\item Calculer les racines du trinôme $g:x\mapsto 2x^2-12x+16$
\item En déduire le tableau de signe de $g$
\item En déduire finalement le domaine de définition de $f$
\item On admet que $g$ est décroissante sur $]-\infty,3[$ et croissante sur $]3,+\infty[$. Etudier ainsi le sens de variation de $x\mapsto \sqrt{g(x)}$
\item Donner le sens de variation de $f$
\item Calculer $f(-1)$ et $f(5)$
\item Représenter graphiquement $f$
\end{enumerate}
\subsection*{Exercice 3 (Valeur absolue, temps conseillé : 10 min) : }
\begin{enumerate}
\item Rappeler la définition de la valeur absolue
\item Montrer que pour tout $a$ et $b$ réels, 
$$|a\times b| = |a| \times |b|$$
\item On considère la fonction $h$ définie par $h:x\mapsto x^2-4x+4$. Montrer que pour tout $x\in \R$, $$|h(x)| = h(x)$$\emph{Indication : on pourra utiliser la forme factorisée de $f$}
\end{enumerate}
\subsection*{Exercice 4 (Intersection de deux droites, temps conseillé : 17-20 min) : }
On considère les fonctions $$f(x) = ax+b$$$$g(x) = cx+d$$
On s'intéresse à l'intersection de leurs droites représentatives $\mathscr{C}_f$ et $\mathscr{C}_g$
\begin{enumerate}
\item Montrer, en donnant des exemples, qu'il y a trois situations possibles : \begin{itemize} \item Pas de point en commun \item Un unique point en commun \item Une infinité de point en commun \end{itemize}
\item Montrer que si $(x,y)$ est un point d'intersection, alors 
$$(a-c) x = d-b$$
\emph{On rappelle qu'un point $(x_1,y_1)$ appartient à la courbe représentative de $x\mapsto \alpha x + \beta$ si et seulement si $y_1 = \alpha x_1 + \beta$}
\item On suppose tout d'abord $a-c = 0$. \newline
Si $d =b $, que se passe-t-il ? dans quel cas est-on ? comment l'expliquer vous ?\newline
Si $d \neq b$, que se passe-t-il ? dans quel cas est-on ? que dire de $\mathscr{C}_f$ et $\mathscr{C}_g$ ?
\item On suppose maintenant $a-c \neq 0$. Montrer alors que l'unique point d'intersection des deux droites est $$\left(\frac{d-b}{a-c} ,\frac{ad-bc}{a-c}\right)$$
\end{enumerate}
$$\star \star \star$$
\center
FIN DU SUJET
\end{document}