\documentclass{article}[11pt]
\usepackage[frenchb,english]{babel}
\usepackage[T1]{fontenc}
\usepackage[utf8]{inputenc}
\usepackage{amsmath,amssymb,latexsym}
\usepackage{times}
\usepackage{float}
\usepackage[left=2cm,right=2cm,top=2cm,bottom=2cm]{geometry}
\frenchbsetup{StandardLists=true} % � inclure si on utilise \usepackage[french]{babel}
\usepackage{enumitem}
\usepackage{fancyhdr}
\usepackage{mathrsfs}
\usepackage{graphicx}
%\usepackage[Algorithme]{algorithm}
%\usepackage{algorithmic}
\usepackage{tikz}
\usepackage{tabularx}
\usetikzlibrary{shapes}
\pagestyle{fancy}
\newcommand{\tr}[1]{{\vphantom{#1}}^{\mathit t}{#1}} 
\renewcommand\headrulewidth{1pt}
\fancyhead[L]{Cours 1�re S}
\fancyhead[R]{Yoann Pietri}
\newcounter{theoremecounter}[subsection]
\usepackage{titlesec}
\setcounter{secnumdepth}{3}% enl�ve la num�rotation apr�s les sections
%\renewcommand\thechapter {\Roman{chapter}}

 \setlength{\parindent}{0pt}

\newcommand{\R}{\mathbb{R}}
\newcommand{\N}{\mathbb{N}}
\newcommand{\Q}{\mathbb{Q}}
\newcommand{\Z}{\mathbb{Z}}
\newcommand{\C}{\mathbb{C}}
\newcommand{\K}{\mathbb{K}}
\newcommand{\eqi}{\Leftrightarrow}
\titleformat{\subsubsection}
   {\normalfont\fontsize{11pt}{13pt}\selectfont\bfseries}% apparence commune au titre et au num�ro
   {\thesubsubsection}% apparence du num�ro
   {1em}% espacement num�ro/texte
   {}% apparence du titre

\tikzstyle{theobox} = [draw=black, very thick,
    rectangle, rounded corners, inner sep=10pt, inner ysep=20pt]
\tikzstyle{theotitle} =[fill=white, text=black,rounded corners,draw=black,very thick]

\fancyhead[L]{Contrôle chapitre 1}

\begin{document}
\center
\Large Contrôle de cours
\flushleft
\center
Trinômes du second degré 
\flushleft \normalsize
Durée du contrôle : 1h\newline
Ce sujet comporte 2 pages\newline
La calculatrice est autorisée
\subsection*{Exercice 1 (R.O.C., temps conseillé : 10 min) : }
Redémontrer le théorème d'unicité des coefficients puis donner sans démonstration le résultat sur les formes canoniques de trinômes avant de démontrer le résultat sur les racines d'un trinôme.
\subsection*{Exercice 2 (Etude d'une fonction trinôme, temps conseillé : 15 min) : }
On considère la fonction $f$ définie pour tout $x$ par $$f(x) = 2x^2-5x-3$$
\begin{enumerate}
\item Donner l'ensemble de définition de $f$
\item Calculer $f(1)$, $f(4)$ et $f(-1)$
\item Mettre le trinôme sous sa forme canonique
\item Calculer les (ou la) racine(s) (si elle(s) existe(nt)) de ce trinôme
\item Mettre $f$ sous sa forme factorisée (si cela est possible)
\item Etablir le tableau de signe de $f$
\item Tracer la représentation graphique de $f$
\end{enumerate}
\subsection*{Exercice 3 (Equation et inéquation du second degré, temps conseillé : 15 min) : }
\begin{enumerate}
\item Résoudre, dans $\R$, l'équation suivante
$$2x^2 + 7 x -1 = 1 - x^2 + 2x$$
\item Résoudre, dans $\R$, l'inéquation suivante
$$x\left(\frac{1}{2}x + \frac{3}{2}\right) < 2$$
\end{enumerate}
\subsection*{Exercice 4 (Symétries, temps conseillé : 15-20 min) : }
On considère la fonction $f$ définie pour tout $x\in \R$ par 
$$f(x) = ax^2+bx+c$$
avec $a\neq 0$
\begin{enumerate}
\item Calculer $\displaystyle f\left(-\frac{b}{2a}\right)$
\item Calculer $f(x) - f\left(-\frac{b}{2a}\right)$ (on mettra $f$ sous sa forme canonique)
\item En déduire que si $a$ est positif alors $f\left(-\frac{b}{2a}\right)$ est un minimum et que si $a$ est négatif alors c'est un maximum
\item Calculer, pour tout $x\in \R$, $$f\left(-\frac{b}{2a} - x\right)$$ et $$f\left(-\frac{b}{2a} + x\right)$$ Que remarquez vous ? Quelle en est la conséquence sur la courbe représentative de $f$ par rapport à la droite d'équation $x = -\frac{b}{2a}$ ? (aidez vous du titre de l'exercice, et faites le lien avec le cours de seconde)
\item Tracer la courbe représentative de $x\mapsto x^2-x-1$ et la droite d'équation $x=\frac{1}{2}$. Que remarquez-vous ?
\end{enumerate}
$$\star \star \star$$
\center
FIN DU SUJET
\end{document}