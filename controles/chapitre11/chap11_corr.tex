\documentclass{article}[11pt]
\usepackage[frenchb,english]{babel}
\usepackage[T1]{fontenc}
\usepackage[utf8]{inputenc}
\usepackage{amsmath,amssymb,latexsym}
\usepackage{times}
\usepackage{float}
\usepackage[left=2cm,right=2cm,top=2cm,bottom=2cm]{geometry}
\frenchbsetup{StandardLists=true} % � inclure si on utilise \usepackage[french]{babel}
\usepackage{enumitem}
\usepackage{fancyhdr}
\usepackage{mathrsfs}
\usepackage{graphicx}
%\usepackage[Algorithme]{algorithm}
%\usepackage{algorithmic}
\usepackage{tikz}
\usepackage{tabularx}
\usetikzlibrary{shapes}
\pagestyle{fancy}
\newcommand{\tr}[1]{{\vphantom{#1}}^{\mathit t}{#1}} 
\renewcommand\headrulewidth{1pt}
\fancyhead[L]{Cours 1�re S}
\fancyhead[R]{Yoann Pietri}
\newcounter{theoremecounter}[subsection]
\usepackage{titlesec}
\setcounter{secnumdepth}{3}% enl�ve la num�rotation apr�s les sections
%\renewcommand\thechapter {\Roman{chapter}}

 \setlength{\parindent}{0pt}

\newcommand{\R}{\mathbb{R}}
\newcommand{\N}{\mathbb{N}}
\newcommand{\Q}{\mathbb{Q}}
\newcommand{\Z}{\mathbb{Z}}
\newcommand{\C}{\mathbb{C}}
\newcommand{\K}{\mathbb{K}}
\newcommand{\eqi}{\Leftrightarrow}
\titleformat{\subsubsection}
   {\normalfont\fontsize{11pt}{13pt}\selectfont\bfseries}% apparence commune au titre et au num�ro
   {\thesubsubsection}% apparence du num�ro
   {1em}% espacement num�ro/texte
   {}% apparence du titre

\tikzstyle{theobox} = [draw=black, very thick,
    rectangle, rounded corners, inner sep=10pt, inner ysep=20pt]
\tikzstyle{theotitle} =[fill=white, text=black,rounded corners,draw=black,very thick]

\fancyhead[L]{Contrôle chapitre 11}

\usepackage[c]{esvect}
\newcommand{\covec}[2]{\begin{pmatrix}#1 \\#2 \end{pmatrix}}

%\usepackage[latin1]{inputenc}
%\usepackage[T1]{fontenc}
%\usepackage[colorlinks=true,urlcolor=black]{hyperref}
\usepackage{fancyhdr}
%FONTES POUR ALGO
\DeclareFontFamily{T1}{lmtt}{} 
\DeclareFontShape{T1}{lmtt}{m}{n}{<-> ec-lmtl10}{} 
\DeclareFontShape{T1}{lmtt}{m}{\itdefault}{<-> ec-lmtlo10}{} 
\DeclareFontShape{T1}{lmtt}{\bfdefault}{n}{<-> ec-lmtk10}{} 
\DeclareFontShape{T1}{lmtt}{\bfdefault}{\itdefault}{<-> ec-lmtko10}{}
\renewcommand{\ttdefault}{lmtt}
% PACKAGES NECESSAIRES POUR ALGO
\usepackage{xcolor}
\usepackage{framed}
\usepackage{algorithm}
\usepackage{algpseudocode}
%ALGO
\definecolor{fond}{RGB}{136,136,136}
\definecolor{sicolor}{RGB}{128,0,128}
\definecolor{tantquecolor}{RGB}{221,111,6}
\definecolor{pourcolor}{RGB}{187,136,0}
\definecolor{bloccolor}{RGB}{128,0,0}
\newenvironment{cadrecode}{%
\def\FrameCommand{{\color{fond}\vrule width 5pt}\fcolorbox{fond}{white}}%
\MakeFramed {\hsize \linewidth \advance\hsize-\width \FrameRestore}\begin{footnotesize}}%
{\end{footnotesize}\endMakeFramed}
\makeatletter
\def\therule{\makebox[\algorithmicindent][l]{\hspace*{.5em}\color{fond} \vrule width 1pt height .75\baselineskip depth .25\baselineskip}}%
\newtoks\therules
\therules={}
\def\appendto#1#2{\expandafter#1\expandafter{\the#1#2}}
\def\gobblefirst#1{#1\expandafter\expandafter\expandafter{\expandafter\@gobble\the#1}}%
\def\Ligne{\State\unskip\the\therules}% 
\def\pushindent{\appendto\therules\therule}%
\def\popindent{\gobblefirst\therules}%
\def\printindent{\unskip\the\therules}%
\def\printandpush{\printindent\pushindent}%
\def\popandprint{\popindent\printindent}%
\def\Variables{\Ligne \textcolor{bloccolor}{\textbf{VARIABLES}}}
\def\Si#1{\Ligne \textcolor{sicolor}{\textbf{SI}} #1 \textcolor{sicolor}{\textbf{ALORS}}}%
\def\Sinon{\Ligne \textcolor{sicolor}{\textbf{SINON}}}%
\def\Pour#1#2#3{\Ligne \textcolor{pourcolor}{\textbf{POUR}} #1 \textcolor{pourcolor}{\textbf{ALLANT\_DE}} #2 \textcolor{pourcolor}{\textbf{A}} #3}%
\def\Tantque#1{\Ligne \textcolor{tantquecolor}{\textbf{TANT\_QUE}} #1 \textcolor{tantquecolor}{\textbf{FAIRE}}}%
\algdef{SE}[WHILE]{DebutTantQue}{FinTantQue}
  {\pushindent \printindent  \textcolor{tantquecolor}{\textbf{DEBUT\_TANT\_QUE}}}
  {\printindent \popindent  \textcolor{tantquecolor}{\textbf{FIN\_TANT\_QUE}}}%
\algdef{SE}[FOR]{DebutPour}{FinPour}
  {\pushindent \printindent \textcolor{pourcolor}{\textbf{DEBUT\_POUR}}}
  {\printindent \popindent  \textcolor{pourcolor}{\textbf{FIN\_POUR}}}%
\algdef{SE}[IF]{DebutSi}{FinSi}%
  {\pushindent \printindent \textcolor{sicolor}{\textbf{DEBUT\_SI}}}
  {\printindent \popindent \textcolor{sicolor}{\textbf{FIN\_SI}}}%
\algdef{SE}[IF]{DebutSinon}{FinSinon}
  {\pushindent \printindent \textcolor{sicolor}{\textbf{DEBUT\_SINON}}}
  {\printindent \popindent \textcolor{sicolor}{\textbf{FIN\_SINON}}}%
\algdef{SE}[PROCEDURE]{DebutAlgo}{FinAlgo}
   {\printandpush \textcolor{bloccolor}{\textbf{DEBUT\_ALGORITHME}}}%
   {\popandprint \textcolor{bloccolor}{\textbf{FIN\_ALGORITHME}}}%
\makeatother
\newenvironment{algobox}%
{%
\begin{ttfamily}
\begin{algorithmic}[1]
\begin{cadrecode}
\labelwidth 1.5em
\leftmargin\labelwidth \addtolength{\leftmargin}{\labelsep}
}
{%
\end{cadrecode}
\end{algorithmic}
\end{ttfamily}
}



\begin{document}
\center
\Large Contrôle de cours (correction)
\flushleft
\center
Echantillonage
\flushleft \normalsize
\subsection*{Exercice 1 (R.O.C., temps conseillé : 10 min) : }
Voir le cours pour le principe et les objectifs. L'intervalle est alors $$\left[p-\frac{1}{\sqrt{n}},p+\frac{1}{\sqrt{n}}\right]$$
\subsection*{Exercice 2 (temps conseillé : 10 min) : }
On recherche l'intervalle de fluctuation au seuil de 95\% : 
$$\left[0,71-\frac{1}{\sqrt{98765}},0,71+\frac{1}{\sqrt{98765}}\right]$$
$$=[0,706818;0,713182]$$
De plus
$$f=\frac{69234}{98765} = 0,700997$$
$$f\notin[0,706818;0,713182]$$
donc l'échantillon n'est pas représentatif de la population.
\subsection*{Exercice 3 (Algorithmique, temps conseillé : 20 min) : }
\begin{enumerate}
\item Algorithme de la factorielle
\begin{algobox}
\Variables
\Ligne n EST\_DU\_TYPE NOMBRE
\Ligne res EST\_DU\_TYPE NOMBRE
\Ligne k EST\_DU\_TYPE NOMBRE
\DebutAlgo
\Ligne SAISIR n
\Ligne res PREND\_LA\_VALEUR 1
\Pour{k}{2}{n}
\DebutPour
\Ligne res PREND\_LA\_VALEUR res * k
\FinPour
\Ligne AFFICHER res
\FinAlgo

\end{algobox}
\item Algorithme des nombres binomiaux
\begin{algobox}
\Variables
\Ligne n EST\_DU\_TYPE NOMBRE
\Ligne k EST\_DU\_TYPE NOMBRE
\Ligne res EST\_DU\_TYPE NOMBRE
\DebutAlgo
\Ligne SAISIR n
\Ligne SAISIR k
\Ligne res PREND\_LA\_VALEUR factorielle(n)/(factorielle(k)*factorielle(n-k))
\Ligne AFFICHER res
\FinAlgo

\end{algobox}
\item Algorithme de $P(X=k)$
\begin{algobox}
\Variables
\Ligne p EST\_DU\_TYPE NOMBRE
\Ligne n EST\_DU\_TYPE NOMBRE
\Ligne k EST\_DU\_TYPE NOMBRE
\Ligne res EST\_DU\_TYPE NOMBRE
\DebutAlgo
\Ligne SAISIR n
\Ligne SAISIR p
\Ligne SAISIR k
\Ligne res PREND\_LA\_VALEUR binom(k,n)*pow(p,k)*pow(1-p,n-k)
\Ligne AFFICHER res
\FinAlgo
\end{algobox}
\item Algorithme de $P(X\leq k)$
\begin{algobox}
\Variables
\Ligne n EST\_DU\_TYPE NOMBRE
\Ligne p EST\_DU\_TYPE NOMBRE
\Ligne k EST\_DU\_TYPE NOMBRE
\Ligne i EST\_DU\_TYPE NOMBRE
\Ligne res EST\_DU\_TYPE NOMBRE
\DebutAlgo
\Ligne SAISIR n
\Ligne SAISIR p
\Ligne SAISIR k
\Ligne res PREND\_LA\_VALEUR 0
\Pour{i}{0}{k}
\DebutPour
\Ligne res PREND\_LA\_VALEUR res + PBinom(n,p,k)
\FinPour
\Ligne AFFICHER res
\FinAlgo

\end{algobox}
\item Algorithme borne sup
\begin{algobox}
\Variables
\Ligne n EST\_DU\_TYPE NOMBRE
\Ligne p EST\_DU\_TYPE NOMBRE
\Ligne p0 EST\_DU\_TYPE NOMBRE
\Ligne k EST\_DU\_TYPE NOMBRE
\DebutAlgo
\Ligne SAISIR n
\Ligne SAISIR p
\Ligne k PREND\_LA\_VALEUR 0
\Tantque{(PInfBinom(n,p,k) < p0)}
\DebutTantQue
\Ligne k PREND\_LA\_VALEUR k+1
\FinTantQue
\Ligne AFFICHER k
\FinAlgo

\end{algobox}
\item \begin{verbatim} LimSupBinom(n,p,0.975) \end{verbatim} représente la borne supérieur de l'intervalle de fluctuation
\end{enumerate}
\subsection*{Exercice 4 (temps conseillé : 20 min) : }
\begin{enumerate}
\item $X$ suit la loi binomiale de paramètre 100 et $\frac{3}{10}$ $$\boxed{X \sim \mathscr{B}(100,0.3)}$$
\item On cherche $a$ le plus grand entier tel que $P(X \leq a) < 2.5$\% : c'est $a=20$. On cherche $b$ le plus petit entier tel que $P(X \leq b) \geq 97.5$\% C'est $b=39$. Ainsi l'intervalle de fluctuation au seuil de 95\% est $$\left[\frac{21}{100},\frac{39}{100}\right]$$
\item
\begin{verbatim} LimSupBinom(100,0.3,0.975) \end{verbatim} aurait renvoyé $39$
\item Non. C'est dû à des erreurs de calcul de la machine
\item $41 \notin [21,39]$. Il y avait en fait 5\% de chance que Max tire un nombre de cartes qui n'était pas dans $[21,39]$
\end{enumerate}
$$\star \star \star$$
\center
FIN DU SUJET
\end{document}