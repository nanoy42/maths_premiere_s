\documentclass{article}[11pt]
\usepackage[frenchb,english]{babel}
\usepackage[T1]{fontenc}
\usepackage[utf8]{inputenc}
\usepackage{amsmath,amssymb,latexsym}
\usepackage{times}
\usepackage{float}
\usepackage[left=2cm,right=2cm,top=2cm,bottom=2cm]{geometry}
\frenchbsetup{StandardLists=true} % � inclure si on utilise \usepackage[french]{babel}
\usepackage{enumitem}
\usepackage{fancyhdr}
\usepackage{mathrsfs}
\usepackage{graphicx}
%\usepackage[Algorithme]{algorithm}
%\usepackage{algorithmic}
\usepackage{tikz}
\usepackage{tabularx}
\usetikzlibrary{shapes}
\pagestyle{fancy}
\newcommand{\tr}[1]{{\vphantom{#1}}^{\mathit t}{#1}} 
\renewcommand\headrulewidth{1pt}
\fancyhead[L]{Cours 1�re S}
\fancyhead[R]{Yoann Pietri}
\newcounter{theoremecounter}[subsection]
\usepackage{titlesec}
\setcounter{secnumdepth}{3}% enl�ve la num�rotation apr�s les sections
%\renewcommand\thechapter {\Roman{chapter}}

 \setlength{\parindent}{0pt}

\newcommand{\R}{\mathbb{R}}
\newcommand{\N}{\mathbb{N}}
\newcommand{\Q}{\mathbb{Q}}
\newcommand{\Z}{\mathbb{Z}}
\newcommand{\C}{\mathbb{C}}
\newcommand{\K}{\mathbb{K}}
\newcommand{\eqi}{\Leftrightarrow}
\titleformat{\subsubsection}
   {\normalfont\fontsize{11pt}{13pt}\selectfont\bfseries}% apparence commune au titre et au num�ro
   {\thesubsubsection}% apparence du num�ro
   {1em}% espacement num�ro/texte
   {}% apparence du titre

\tikzstyle{theobox} = [draw=black, very thick,
    rectangle, rounded corners, inner sep=10pt, inner ysep=20pt]
\tikzstyle{theotitle} =[fill=white, text=black,rounded corners,draw=black,very thick]

\fancyhead[L]{Contrôle chapitre 8}

\usepackage[c]{esvect}
\newcommand{\covec}[2]{\begin{pmatrix}#1 \\#2 \end{pmatrix}}

\begin{document}
\center
\Large Contrôle de cours
\flushleft
\center
Statistiques
\flushleft \normalsize
Durée du contrôle : 1h\newline
Ce sujet comporte 2 pages\newline
La calculatrice est autorisée
\subsection*{Exercice 1 (R.O.C., temps conseillé : 10 min) : }
Après avoir rappelé la formule donnant la variance d'une série statistique, vous montrerez la formule de König-Huygens
\subsection*{Exercice 2 (Caractère discret, temps conseillé : 15 min) : }
Lors d'un devoir commun d'élèves hétérogènes, évaluant 10 capacités, on a les résultats suivants\newline

\begin{tabularx}{\linewidth}{|X|X|X|X|X|X|X|X|X|X|X|X|}
\hline
N (*) & 0 & 1 & 2 & 3 & 4 & 5 & 6 & 7 & 8 & 9 & 10\\ \hline
Effectif & 10 & 8 & 7 & 3 & 2 & 3 & 5 &9 & 11 & 13 & 15 \\ \hline
\end{tabularx}\newline

(*) : Nombre de capacités validées
\begin{enumerate}
\item Pourquoi parle t-on de caractère discret ? Quel est la différence avec un caractère continu ?
\item Calculer l'effectif de cette série statistique
\item Mettre dans un tableau les fréquences
\item Calculer la moyenne, la médiane, les quartiles, l'écart inter-quartiles, l'étendue, la variance et l'écart-type de cette série statistique. Interpréter chacune des valeurs obtenues et leur utilité. 
\item Représenter l'histogramme et la boîte à moustache de cette série statistique
\item Pourquoi parle-ton d'élèves hétérogènes \emph{(On attend ici un raisonnement basé sur (médiane, écart-interquartiles) ou sur (moyenne, écart-type) ou sur les deux)}
\end{enumerate}
\subsection*{Exercice 3 (Caractère continu, temps conseillé : 15 min) : }
Le ministère de la jeunesse décide de faire une enquête sur le temps passé sur internet par des jeunes. Ils font d'abord une enquête sur les 13-18 ans puis sur les 18-25 ans. C'est Max, le stagiaire qui a recopié les chiffres et il a fait un peu n'importe quoi : \newline


\textbf{1ère série (13-18 ans)}\newline


Effectif : 14803\newline


\begin{tabularx}{\linewidth}{|X|X|X|X|X|X|X|X|X|X|X|X|X|X|X|X|}
\hline
T (*) & 0-2h  & 2-4h & 4-6h & 6-8h & 8-10h & 10-12h & 12-14h & 14-16h & 16-18h & 18-20h & 20-22h & 22-24h & 24-26h & 26-28h & 28-30h\\ \hline     
Effec- tif & 78 & 100 & 254 & 563 & 1998 & 4257 & 3048 & 1650 & 1150 & 590 & 500 & ..... & 187 & 121 & 95 \\ \hline
\end{tabularx}\newline


(*) temps passé, en heures, par semaine\newline

\textbf{2ème série (18-25 ans)}\newline

Effectif : \newline

\begin{tabularx}{\linewidth}{|X|X|X|X|X|X|X|X|X|}
\hline
T (*) & 0-2h  & 2-4h & 4-6h & 6-8h & 8-10h & 10-12h & 12-14h & 14-16h\\ \hline     
Fréquence & 0,0011 & 0,0028 & 0,0062 & 0,0222 & 0,0393 & 0,0638 & 0,1269 & 0,1657\\ \hline
\end{tabularx}\newline
\begin{tabularx}{\linewidth}{|X|X|X|X|X|X|X|X|X|}
\hline
T (*)  & 16-18h & 18-20h & 20-22h & 22-24h & 24-26h & 26-28h & 28-30h & RIEN ICI\\ \hline     
Fréquence  & 0,2712 & 0,1491 & ..... & 0,0434 & 0,0266 & 0,0110 & 0,0066 & RIEN ICI \\ \hline
\end{tabularx}\newline

(*) temps passé, en heures, par semaine\newline
\begin{enumerate}
\item Compléter les données manquantes (2 cas à compléter)
\item Raconter tout ce que vous pouvez sur la première série statistique (moyenne, médiane, étendue, quartiles, écart-type, variance, écart interquartiles, histogramme, diagramme en boites,...). Interpréter
\item Calculer la moyenne, la variance et l'écart type de la deuxième série statistique. Interpréter.
\item Comparer les résultats
\end{enumerate}
\subsection*{Exercice 4 (Quelques nouveaux outils, temps conseillé : 20 min) : }
On définit (en reprenant les mêmes notations du cours) : 
\begin{itemize}
\item La moyenne géométrique d'une série statistique 
$$\overset{\circ}{x} = \sqrt[n]{\prod_{i=1}^p f_i x_i}$$
\item L'écart arithmétique 
$$\overset{\sim}{e} = \frac{1}{n}\sum_{i=1}^p |x_i - \overline{x}|$$
\item Le critère de dispersion 
$$C_V = \frac{\sigma}{\overline{x}}\times 100$$
\item Coefficient d'asymétrie de Pearson 
$$\lambda_x = \frac{\overline{x} - Q_2}{\sigma}$$
\item Le coefficient de Yule 
$$\mathscr{Y}_x = \frac{Q_3+Q_1-2Q_1}{Q_3-Q_1}$$
\item Coefficient d'aplatissement de Fischer
$$\mathscr{F}_x = \frac{1}{n}\frac{\displaystyle \sum_{i=1}^p (x_i - \overline{x})^4}{\sigma^4}$$
\end{itemize}
\begin{enumerate}
\item Calculer leur valeur pour la série statistique de l'exercice 1
\item Essayer d'interpréter chacun des termes rapidement
\item Montrer que 
$$\sqrt{ab} \leq \frac{a+b}{2}$$
pour tout $a$ et $b$ réels.
\emph{(Indication : après avoir remarqué que $(a-b)^2 \geq 0$, développer et ajouter $4ab$ des deux côtés pour conclure)}
Ce qu'on vient de faire permet, au prix d'une récurrence de montrer 
$$\overset{\circ}{x} \leq \overline{x}$$
\end{enumerate}
$$\star \star \star$$
\center
FIN DU SUJET
\end{document}