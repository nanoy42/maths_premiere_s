\documentclass{article}[11pt]
\usepackage[frenchb,english]{babel}
\usepackage[T1]{fontenc}
\usepackage[utf8]{inputenc}
\usepackage{amsmath,amssymb,latexsym}
\usepackage{times}
\usepackage{float}
\usepackage[left=2cm,right=2cm,top=2cm,bottom=2cm]{geometry}
\frenchbsetup{StandardLists=true} % � inclure si on utilise \usepackage[french]{babel}
\usepackage{enumitem}
\usepackage{fancyhdr}
\usepackage{mathrsfs}
\usepackage{graphicx}
%\usepackage[Algorithme]{algorithm}
%\usepackage{algorithmic}
\usepackage{tikz}
\usepackage{tabularx}
\usetikzlibrary{shapes}
\pagestyle{fancy}
\newcommand{\tr}[1]{{\vphantom{#1}}^{\mathit t}{#1}} 
\renewcommand\headrulewidth{1pt}
\fancyhead[L]{Cours 1�re S}
\fancyhead[R]{Yoann Pietri}
\newcounter{theoremecounter}[subsection]
\usepackage{titlesec}
\setcounter{secnumdepth}{3}% enl�ve la num�rotation apr�s les sections
%\renewcommand\thechapter {\Roman{chapter}}

 \setlength{\parindent}{0pt}

\newcommand{\R}{\mathbb{R}}
\newcommand{\N}{\mathbb{N}}
\newcommand{\Q}{\mathbb{Q}}
\newcommand{\Z}{\mathbb{Z}}
\newcommand{\C}{\mathbb{C}}
\newcommand{\K}{\mathbb{K}}
\newcommand{\eqi}{\Leftrightarrow}
\titleformat{\subsubsection}
   {\normalfont\fontsize{11pt}{13pt}\selectfont\bfseries}% apparence commune au titre et au num�ro
   {\thesubsubsection}% apparence du num�ro
   {1em}% espacement num�ro/texte
   {}% apparence du titre

\tikzstyle{theobox} = [draw=black, very thick,
    rectangle, rounded corners, inner sep=10pt, inner ysep=20pt]
\tikzstyle{theotitle} =[fill=white, text=black,rounded corners,draw=black,very thick]

\fancyhead[L]{Contrôle chapitre 4}

\usepackage{tcolorbox}

%\usepackage[latin1]{inputenc}
%\usepackage[T1]{fontenc}
%\usepackage[colorlinks=true,urlcolor=black]{hyperref}
\usepackage{fancyhdr}
%FONTES POUR ALGO
\DeclareFontFamily{T1}{lmtt}{} 
\DeclareFontShape{T1}{lmtt}{m}{n}{<-> ec-lmtl10}{} 
\DeclareFontShape{T1}{lmtt}{m}{\itdefault}{<-> ec-lmtlo10}{} 
\DeclareFontShape{T1}{lmtt}{\bfdefault}{n}{<-> ec-lmtk10}{} 
\DeclareFontShape{T1}{lmtt}{\bfdefault}{\itdefault}{<-> ec-lmtko10}{}
\renewcommand{\ttdefault}{lmtt}
% PACKAGES NECESSAIRES POUR ALGO
\usepackage{xcolor}
\usepackage{framed}
\usepackage{algorithm}
\usepackage{algpseudocode}
%ALGO
\definecolor{fond}{RGB}{136,136,136}
\definecolor{sicolor}{RGB}{128,0,128}
\definecolor{tantquecolor}{RGB}{221,111,6}
\definecolor{pourcolor}{RGB}{187,136,0}
\definecolor{bloccolor}{RGB}{128,0,0}
\newenvironment{cadrecode}{%
\def\FrameCommand{{\color{fond}\vrule width 5pt}\fcolorbox{fond}{white}}%
\MakeFramed {\hsize \linewidth \advance\hsize-\width \FrameRestore}\begin{footnotesize}}%
{\end{footnotesize}\endMakeFramed}
\makeatletter
\def\therule{\makebox[\algorithmicindent][l]{\hspace*{.5em}\color{fond} \vrule width 1pt height .75\baselineskip depth .25\baselineskip}}%
\newtoks\therules
\therules={}
\def\appendto#1#2{\expandafter#1\expandafter{\the#1#2}}
\def\gobblefirst#1{#1\expandafter\expandafter\expandafter{\expandafter\@gobble\the#1}}%
\def\Ligne{\State\unskip\the\therules}% 
\def\pushindent{\appendto\therules\therule}%
\def\popindent{\gobblefirst\therules}%
\def\printindent{\unskip\the\therules}%
\def\printandpush{\printindent\pushindent}%
\def\popandprint{\popindent\printindent}%
\def\Variables{\Ligne \textcolor{bloccolor}{\textbf{VARIABLES}}}
\def\Si#1{\Ligne \textcolor{sicolor}{\textbf{SI}} #1 \textcolor{sicolor}{\textbf{ALORS}}}%
\def\Sinon{\Ligne \textcolor{sicolor}{\textbf{SINON}}}%
\def\Pour#1#2#3{\Ligne \textcolor{pourcolor}{\textbf{POUR}} #1 \textcolor{pourcolor}{\textbf{ALLANT\_DE}} #2 \textcolor{pourcolor}{\textbf{A}} #3}%
\def\Tantque#1{\Ligne \textcolor{tantquecolor}{\textbf{TANT\_QUE}} #1 \textcolor{tantquecolor}{\textbf{FAIRE}}}%
\algdef{SE}[WHILE]{DebutTantQue}{FinTantQue}
  {\pushindent \printindent  \textcolor{tantquecolor}{\textbf{DEBUT\_TANT\_QUE}}}
  {\printindent \popindent  \textcolor{tantquecolor}{\textbf{FIN\_TANT\_QUE}}}%
\algdef{SE}[FOR]{DebutPour}{FinPour}
  {\pushindent \printindent \textcolor{pourcolor}{\textbf{DEBUT\_POUR}}}
  {\printindent \popindent  \textcolor{pourcolor}{\textbf{FIN\_POUR}}}%
\algdef{SE}[IF]{DebutSi}{FinSi}%
  {\pushindent \printindent \textcolor{sicolor}{\textbf{DEBUT\_SI}}}
  {\printindent \popindent \textcolor{sicolor}{\textbf{FIN\_SI}}}%
\algdef{SE}[IF]{DebutSinon}{FinSinon}
  {\pushindent \printindent \textcolor{sicolor}{\textbf{DEBUT\_SINON}}}
  {\printindent \popindent \textcolor{sicolor}{\textbf{FIN\_SINON}}}%
\algdef{SE}[PROCEDURE]{DebutAlgo}{FinAlgo}
   {\printandpush \textcolor{bloccolor}{\textbf{DEBUT\_ALGORITHME}}}%
   {\popandprint \textcolor{bloccolor}{\textbf{FIN\_ALGORITHME}}}%
\makeatother
\newenvironment{algobox}%
{%
\begin{ttfamily}
\begin{algorithmic}[1]
\begin{cadrecode}
\labelwidth 1.5em
\leftmargin\labelwidth \addtolength{\leftmargin}{\labelsep}
}
{%
\end{cadrecode}
\end{algorithmic}
\end{ttfamily}
}



\begin{document}
\center
\Large Contrôle de cours (correction)
\flushleft
\center
Suites
\flushleft \normalsize
\subsection*{Exercice 1 (R.O.C., temps conseillé : 10 min) : }
On peut générer une suite de 3 manières : explicitement (exemple pour tout $n$, $u_n = n$), par récurrence : $$\left\{\begin{array}{l} u_0 = 1 \\ \forall n \in \N \quad u_{n+1} = \sqrt{u_n}\end{array}\right.$$ et implicitement

On appelle suite arithmétique de raison $r$ et de premier terme $u_0$ la suite définie par $$\left\{\begin{array}{l} u_0 = u_0 \\ \forall n \in \N \quad u_{n+1} = u_n + r\end{array}\right.$$On a alors $$u_n = u_{n-1} + r = u_{n-2} + 2r = \ldots = u0 + nr$$ De plus $u_{n+1} -u_n = r$ pour tout $n$ dans $\N$ donc $(u_n)$ est strictement croissante si $r>0$, constante si $r=0$ et strictement décroissante si $r< 0$
\subsection*{Exercice 2 (Etude d'une suite, temps conseillé : 20 min) : }
\begin{enumerate}
\item A $t = n min$, la température de l'eau est $u_n$ et elle va prendre 2 degrés Celsius avant d'arriver $t = n+1 min$. Ainsi la température à $t = n+1 min$ est de 2 degrés plus élevée que celle à $t=n min$ : $$\boxed{u_{n+1} = u_n +2}$$ La température de l'eau étant initialement de $25$ degrés Celsius, on a $$\boxed{u_0 = 25}$$ $(u_n)$ est la suite arithmétique de raison 2 et de premier terme 25
\item On déduit que pour tout $n\in \N$, 
$$\boxed{u_n = 25 + 2n}$$
\item Suite arithmétique de raison positive don c$(u_n)$ est (strictement) croissante
\item On a calculé les 10 premiers termes : \newline
\begin{tabularx}{\linewidth}{|X|X|X|X|X|X|X|X|X|X|X|}
\hline
$n$ & 0 & 1 & 2 & 3 & 4 & 5 & 6 & 7 & 8 & 9 \\ \hline
$u_n$ & 25 & 27 & 29 & 31 & 33 & 35 & 37 & 39 & 41 & 43 \\ \hline 
\end{tabularx}\newline 
puis on les a placé sur un graphe...\newline
\includegraphics[scale=0.2]{chap4_corr_ill1.png}

\item Voici un algorithme qui convient \newline
\begin{algobox}
\Variables
\Ligne u EST\_DU\_TYPE NOMBRE
\Ligne A EST\_DU\_TYPE NOMBRE
\Ligne N EST\_DU\_TYPE NOMBRE
\DebutAlgo
\Ligne SAISIR A
\Ligne u PREND\_LA\_VALEUR 25
\Ligne N PREND\_LA\_VALEUR 0
\Tantque{(u < A)}
\DebutTantQue
\Ligne N PREND\_LA\_VALEUR N+1
\Ligne u PREND\_LA\_VALEUR u+2
\FinTantQue
\Ligne AFFICHER N
\FinAlgo
\end{algobox}
\item On a
$$u_n \geq 100$$
$$\Leftrightarrow 25 + 2n \geq 100$$
$$\Leftrightarrow 2n \geq 75$$
$$\Leftrightarrow n \geq 37,5$$
\item Au bout de 38 minutes (ou 37 minutes et 30 secondes si l'on suppose la chauffe linéaire entre les minutes $n$ et $n+1$)
\item Mauvaise car la température tendrait vers l'infini 
\end{enumerate}
\subsection*{Exercice 3 (Limite d'une suite géométrique, temps conseillé : 20 min) : }
\begin{enumerate}
\item On appelle suite arithmétique de raison $q$ et de premier terme $v_0$ la suite définie par $$\left\{\begin{array}{l} v_0 = v_0 \\ \forall n \in \N \quad v_{n+1} = qv_n\end{array}\right.$$
On a alors pour tout $n$, $$\boxed{v_n = v_0 q^n}$$
\item Pour tout $n\in \N$, $v_n = v_0\times1^n = v_0$
\item 
$$v_n \geq A$$
$$v_0 q^n \geq A$$
$$q^n \geq \frac{A}{v_0}$$
$$\ln(q^n) \geq \ln\left(\frac{A}{v_0}\right)$$
$$n\ln(q) \geq \ln\left(\frac{A}{v_0}\right)$$
$$\boxed{n\geq \frac{\ln(\frac{A}{v_0})}{\ln(q)}}$$
(on ne change pas le sens de l'inégalité car $\ln(q) > 0$ car $q>1$). Pour tout $A$, il existe $N$ tel que pour tout $n \geq N$, $u_n \geq A$ donc $$\boxed{v_n \underset{n\rightarrow \infty}{\longrightarrow} +\infty}$$
\item
$$v_n \leq \varepsilon$$
$$v_0 q^n \leq \varepsilon$$
$$q^n \leq \frac{\varepsilon}{v_0}$$
$$\ln(q^n) \leq \ln\left(\frac{\varepsilon}{v_0}\right)$$
$$n\ln(q) \leq \ln\left(\frac{\varepsilon}{v_0}\right)$$
$$\boxed{n \geq \frac{\ln\left(\frac{\varepsilon}{v_0}\right)}{\ln(q)}}$$
On inverse le sens de l'inégalité car $\ln(q) < 0$. Ainsi tout intervalle ouvert centré en 0 contient toutes les valeurs de la suite à partir d'un certain rang donc 
$$\boxed{v_n \underset{n\rightarrow \infty}{\longrightarrow} 0}$$
\end{enumerate}
\subsection*{Exercice 4 (Exercices classiques du bac (abordable en première), temps conseillé : 20 min) : }
\begin{enumerate}
\item On a $$\boxed{u_0 = 100}$$
$$u_1 =  \frac{3}{4}u_0 + 6 =  \frac{3}{4}\times 100 + 6 = 75 + 6$$
$$\boxed{u_1 = 81}$$
$$u_2 =  \frac{3}{4}u_1 + 6 =  \frac{3}{4}\times 81 + 6 = \frac{243}{4} + 6 = \frac{243}{4} + \frac{24}{4}$$
$$\boxed{u_2 = \frac{267}{4}}$$
$$u_3 =  \frac{3}{4}u_2 + 6 =  \frac{3}{4}\times \frac{267}{4} + 6$$
$$\boxed{u_3 = \frac{897}{16}}$$
\item On a $$v_0 = u_0 - 24$$
$$v_0 = 100 -24$$
$$\boxed{v_0 = 76}$$
De plus 
$$\frac{v_{n+1}}{v_n} = \frac{u_{n+1} - 24}{u_n -24}$$
$$\frac{v_{n+1}}{v_n} = \frac{\frac{3}{4}u_n + 6 -24}{u_n-24}$$
$$\frac{v_{n+1}}{v_n} = \frac{\frac{3}{4}u_n -18}{u_n-24}$$
$$\frac{v_{n+1}}{v_n} = \frac{3}{4}\frac{u_n -\frac{4}{3}18}{u_n-24}$$
$$\frac{v_{n+1}}{v_n} = \frac{3}{4}\frac{u_n -24}{u_n-24}$$
$$\boxed{\frac{v_{n+1}}{v_n} = \frac{3}{4}}$$
\item Pour tout $n\in \N$, on a ainsi 
$$v_n = v_0 \left(\frac{3}{4}\right)^n =  76 \times \left(\frac{3}{4}\right)^n$$
or 
$$v_n = u_n -24$$
donc 
$$u_n = v_n +24$$
d'où
$$\boxed{24 + 76 \times  \left(\frac{3}{4}\right)^n}$$
\item $(v_n)$ est décroissante car $(v_n)$ est une suite géométrique de raison positive inférieure à 1 ($\displaystyle \frac{v_{n+1}}{v_n} = \frac{3}{4}\leq 1$) et $(u_n)$ est décroissante car $u_n = v_n +24$ et $v_n$ décroissante ($u_{n+1} - u_n = v_{n+1} + 24 - (v_n +24) = v_{n+1} - v_n \leq 0$)
\item $v_n$ est une suite géométrique dont la raison est comprise entre 0 et 1 (strictement) donc $$v_n \underset{n\rightarrow \infty}{\longrightarrow} 0$$Ainsi par passage à la limite dans $$u_n = v_n +24$$, on obtient, $$\boxed{u_n \underset{n\rightarrow \infty}{\longrightarrow} 24}$$
\end{enumerate}
$$\star \star \star$$
\center
FIN DU SUJET
\end{document}