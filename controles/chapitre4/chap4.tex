\documentclass{article}[11pt]
\usepackage[frenchb,english]{babel}
\usepackage[T1]{fontenc}
\usepackage[utf8]{inputenc}
\usepackage{amsmath,amssymb,latexsym}
\usepackage{times}
\usepackage{float}
\usepackage[left=2cm,right=2cm,top=2cm,bottom=2cm]{geometry}
\frenchbsetup{StandardLists=true} % � inclure si on utilise \usepackage[french]{babel}
\usepackage{enumitem}
\usepackage{fancyhdr}
\usepackage{mathrsfs}
\usepackage{graphicx}
%\usepackage[Algorithme]{algorithm}
%\usepackage{algorithmic}
\usepackage{tikz}
\usepackage{tabularx}
\usetikzlibrary{shapes}
\pagestyle{fancy}
\newcommand{\tr}[1]{{\vphantom{#1}}^{\mathit t}{#1}} 
\renewcommand\headrulewidth{1pt}
\fancyhead[L]{Cours 1�re S}
\fancyhead[R]{Yoann Pietri}
\newcounter{theoremecounter}[subsection]
\usepackage{titlesec}
\setcounter{secnumdepth}{3}% enl�ve la num�rotation apr�s les sections
%\renewcommand\thechapter {\Roman{chapter}}

 \setlength{\parindent}{0pt}

\newcommand{\R}{\mathbb{R}}
\newcommand{\N}{\mathbb{N}}
\newcommand{\Q}{\mathbb{Q}}
\newcommand{\Z}{\mathbb{Z}}
\newcommand{\C}{\mathbb{C}}
\newcommand{\K}{\mathbb{K}}
\newcommand{\eqi}{\Leftrightarrow}
\titleformat{\subsubsection}
   {\normalfont\fontsize{11pt}{13pt}\selectfont\bfseries}% apparence commune au titre et au num�ro
   {\thesubsubsection}% apparence du num�ro
   {1em}% espacement num�ro/texte
   {}% apparence du titre

\tikzstyle{theobox} = [draw=black, very thick,
    rectangle, rounded corners, inner sep=10pt, inner ysep=20pt]
\tikzstyle{theotitle} =[fill=white, text=black,rounded corners,draw=black,very thick]

\fancyhead[L]{Contrôle chapitre 4}

\usepackage{tcolorbox}


\begin{document}
\center
\Large Contrôle de cours
\flushleft
\center
Suites
\flushleft \normalsize
Durée du contrôle : 1h (il faudra compter un peu plus...)\newline
Ce sujet comporte 2 pages\newline
La calculatrice est autorisée
\subsection*{Exercice 1 (R.O.C., temps conseillé : 10 min) : }
Rappeler les 3 modes de génération d'une suite et donner des exemples pour deux d'entre eux\newline

Rappeler la définition d'une suite arithmétique, redémontrer la formule donnant le terme général d'une suite arithmétique et redémontrer le résultat sur les variations d'une suite arithmétique.
\subsection*{Exercice 2 (Etude d'une suite, temps conseillé : 20 min) : }
Max met de l'eau, initialement à 20 degrés Celsius, à chauffer pour se faire des pâtes. On suppose que l'eau chauffe de 2 degrés Celsius toute les minutes après la mise en chauffe. On note $u_n$ la température de l'eau, en degré Celsius, après $n$ minutes de chauffe
\begin{enumerate}
\item Montrer que $(u_n)$ est une suite arithmétique dont on précisera la raison et le premier terme
\item En déduire le terme général de la suite $(u_n)$
\item Montrer que $(u_n)$ est croissante
\item Calculer les 10 premiers termes de $(u_n)$ et les placer sur un graphe
\item Ecrire un algorithme qui renvoie le premier $N$ tel que pour tout $n\geq N$, $u_n \geq A$ ($A$ saisi par l'utilisateur)
\item Résoudre l'inéquation 
$$u_n \geq 100$$
\item Au bout de combien de temps, Max peut-il mettre ses pâtes ?
\item Que pensez vous de cette modélisation ? \emph{On pourra faire intervenir la notion de limite}
\end{enumerate}
\subsection*{Exercice 3 (Limite d'une suite géométrique, temps conseillé : 20 min) : }
On considère $(v_n)$ une suite géométrique de raison $q > 0$ et de premier terme $v_0 > 0$
\begin{enumerate}
\item Rappeler la définition d'une suite géométrique. Donner aussi le terme général de $(v_n)$
\item Montrer que si $q=1$, alors la suite $(v_n)$ est constante égale à $v_0$\newline
\textbf{Deux questions difficiles arrivent ... Si le candidat a des difficultés à réussir les questions, il est invité à admettre le résultat encadré pour l'exercice 4}\newline

\textbf{On admet l'existence d'une fonction définie sur $\bf{]0,+\infty[}$ appelée logarithme népérien, notée $\bf{\ln}$ qui vérifie : \begin{itemize} \item $\bf{\ln}$ est croissante \item $\bf{\ln(1)=0}$ \item Pour tout $\bf{a\in ]0,+\infty[}$ et $\bf{n \in \N}$, $\bf{\ln(a^n) = n\ln(a)}$ \item $\bf{a > 1 \Leftrightarrow \ln(a) > 0}$ \item $\bf{0 <a < 1 \Leftrightarrow \ln(a) < 0}$ \end{itemize}}
\item On suppose $q > 1$. Soit $A > 0$. Résoudre l'inéquation $$v_n \geq A$$ \emph{(Indication : On pourra (devra...) utiliser le logarithme népérien et donner la solution sous la forme $n \geq \frac{\ln(B)}{\ln(C)}$ où $B$ et $C$ sont à exprimer en fonction de $A$, $v_0$ et $q$)} \newline Que dire de la limite de $(v_n)$ dans ce cas ?
\item On suppose maintenant $q < 1$. Soit $\varepsilon > 0$. On veut savoir à partir de quel rang $[-\varepsilon,\varepsilon]$ contient toutes les valeurs de la suite. Résoudre l'inéquation $$v_n \leq \varepsilon$$  \emph{(Indication : On pourra (devra...) utiliser le logarithme népérien et donner la solution sous la forme $n \geq \frac{\ln(B)}{\ln(C)}$ où $B$ et $C$ sont à exprimer en fonction de $A$, $v_0$ et $q$)}
\end{enumerate}
Quelque soient les réponses apportées aux questions précédentes, on admettra la résultat suivant 
\begin{tcolorbox}Si $(v_n)$ est une suite géométrique de raison $0 < q < 1$ avec $u_0 > 0$ alors $$v_n \underset{n\rightarrow \infty}{\longrightarrow} 0$$\end{tcolorbox}
\subsection*{Exercice 4 (Exercices classiques du bac (abordable en première), temps conseillé : 20 min) : }
On considère la suite définie par 
$$\left\{\begin{array}{l}u_0 = 100 \\\forall n \in \N, \quad u_{n+1} = \frac{3}{4}u_n + 6\end{array}\right.$$
\begin{enumerate}
\item Calculer $u_0$, $u_1$, $u_2$ et $u_3$
\item On considère la suite définie pour tout $n\in \N$ par $$v_n = u_n - 24$$ Montrer que $(v_n)$ est une suite géométrique de raison $\frac{3}{4}$ de d'un premier terme que l'on précisera 
\item Etablir, que pour tout $n \in \N$, 
$$u_n = 24 + 76 \times \left(\frac{3}{4}\right)^n$$
\item Montrer que $(v_n)$ et $(u_n)$ sont décroissantes
\item Montrer que $v_n \underset{n\rightarrow \infty}{\longrightarrow} 0$ \emph{(Indication : utiliser l'exercice 3)}. En déduire la limite de $(u_n)$ lorsque $n$ tend vers l'infini.
\end{enumerate}
$$\star \star \star$$
\center
FIN DU SUJET
\flushleft
\emph{Ce genre de suite est très fréquent au bac : l'utilisation de suite intermédiaire et les $u_n = a + b\times q^n$ tombent (au moins) une fois sur deux}
\end{document}