\documentclass{article}[11pt]
\usepackage[frenchb,english]{babel}
\usepackage[T1]{fontenc}
\usepackage[utf8]{inputenc}
\usepackage{amsmath,amssymb,latexsym}
\usepackage{times}
\usepackage{float}
\usepackage[left=2cm,right=2cm,top=2cm,bottom=2cm]{geometry}
\frenchbsetup{StandardLists=true} % � inclure si on utilise \usepackage[french]{babel}
\usepackage{enumitem}
\usepackage{fancyhdr}
\usepackage{mathrsfs}
\usepackage{graphicx}
%\usepackage[Algorithme]{algorithm}
%\usepackage{algorithmic}
\usepackage{tikz}
\usepackage{tabularx}
\usetikzlibrary{shapes}
\pagestyle{fancy}
\newcommand{\tr}[1]{{\vphantom{#1}}^{\mathit t}{#1}} 
\renewcommand\headrulewidth{1pt}
\fancyhead[L]{Cours 1�re S}
\fancyhead[R]{Yoann Pietri}
\newcounter{theoremecounter}[subsection]
\usepackage{titlesec}
\setcounter{secnumdepth}{3}% enl�ve la num�rotation apr�s les sections
%\renewcommand\thechapter {\Roman{chapter}}

 \setlength{\parindent}{0pt}

\newcommand{\R}{\mathbb{R}}
\newcommand{\N}{\mathbb{N}}
\newcommand{\Q}{\mathbb{Q}}
\newcommand{\Z}{\mathbb{Z}}
\newcommand{\C}{\mathbb{C}}
\newcommand{\K}{\mathbb{K}}
\newcommand{\eqi}{\Leftrightarrow}
\titleformat{\subsubsection}
   {\normalfont\fontsize{11pt}{13pt}\selectfont\bfseries}% apparence commune au titre et au num�ro
   {\thesubsubsection}% apparence du num�ro
   {1em}% espacement num�ro/texte
   {}% apparence du titre

\tikzstyle{theobox} = [draw=black, very thick,
    rectangle, rounded corners, inner sep=10pt, inner ysep=20pt]
\tikzstyle{theotitle} =[fill=white, text=black,rounded corners,draw=black,very thick]

\fancyhead[L]{Contrôle chapitre 3}

\begin{document}
\center
\Large Contrôle de cours
\flushleft
\center
Dérivation
\flushleft \normalsize
Durée du contrôle : 1h\newline
Ce sujet comporte 2 pages\newline
La calculatrice est autorisée
\subsection*{Exercice 1 (R.O.C., temps conseillé : 10 min) : }
Rappeler la définition de la dérivabilité en un point et de la valeur de la dérivée en un point (si elle existe). Rappeler ensuite les formules de dérivée d'une somme, d'un produit et d'un quotient, puis le lien entre extrema et zéros de la dérivée
\subsection*{Exercice 2 (Calculs de dérivée, temps conseillé : 10 min) : }
Pour les 3 fonctions : donner le domaine de définition et montrer qu'elles sont dérivables sur un ensemble à préciser. Exprimer ensuite la dérivée
\begin{enumerate}
\item $$f:x\mapsto 3x^4+4x^2+3$$
\item $$g:x\mapsto \frac{x^3 + \sqrt{x}}{(x+4)^2}$$
\item $$h:x\mapsto \sqrt{x^2-2x+1}$$ sur $[1,+\infty[$
\end{enumerate}
\subsection*{Exercice 3 (Etude de deux fonctions, temps conseillé : 20 min) : }
On définit les fonctions $f$ et $g$ par
$$f:x\mapsto x^3 - \frac{3}{2}x^2 -6x +2$$
$$g:x \mapsto \frac{1}{x^2+1}$$
\begin{enumerate}
\item Donner l'ensemble de définition de $f$
\item Montrer que $f$ est dérivable sur son ensemble de définition
\item Donner sa dérivée $f'$
\item Etablir le tableau de signe de $f'$
\item En déduire le tableau de variation de $f$
\item Rechercher les extrema de $f$
\item Tracer la courbe représentative de $f$
\item Donner l'ensemble de définition de $g$
\item Donner le tableau de variation de $g$ et trouver ses extrema
\end{enumerate}
\subsection*{Exercice 4 (Tableau de variation d'un trinôme, temps conseillé : 20 min) : }
On considère le trinôme défini par $$f:x\mapsto ax^2+bx+c$$
avec $a\neq0$
\begin{enumerate}
\item Montrer que $f$ est dérivable et donner sa dérivée. Comment appelle-t-on ce type de fonction ?
\item Etablir le tableau de variation de $f$ (on distinguera les cas $a>0$ et $a<0$)
\item Montrer que $f$ admet un maximum (resp. un minimum) en $x=-\frac{b}{2a}$ dans le cas $a>0$ (resp. $a<0$)
\end{enumerate}
$$\star \star \star$$
\center
FIN DU SUJET
\end{document}