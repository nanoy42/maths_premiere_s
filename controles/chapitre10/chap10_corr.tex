\documentclass{article}[11pt]
\usepackage[frenchb,english]{babel}
\usepackage[T1]{fontenc}
\usepackage[utf8]{inputenc}
\usepackage{amsmath,amssymb,latexsym}
\usepackage{times}
\usepackage{float}
\usepackage[left=2cm,right=2cm,top=2cm,bottom=2cm]{geometry}
\frenchbsetup{StandardLists=true} % � inclure si on utilise \usepackage[french]{babel}
\usepackage{enumitem}
\usepackage{fancyhdr}
\usepackage{mathrsfs}
\usepackage{graphicx}
%\usepackage[Algorithme]{algorithm}
%\usepackage{algorithmic}
\usepackage{tikz}
\usepackage{tabularx}
\usetikzlibrary{shapes}
\pagestyle{fancy}
\newcommand{\tr}[1]{{\vphantom{#1}}^{\mathit t}{#1}} 
\renewcommand\headrulewidth{1pt}
\fancyhead[L]{Cours 1�re S}
\fancyhead[R]{Yoann Pietri}
\newcounter{theoremecounter}[subsection]
\usepackage{titlesec}
\setcounter{secnumdepth}{3}% enl�ve la num�rotation apr�s les sections
%\renewcommand\thechapter {\Roman{chapter}}

 \setlength{\parindent}{0pt}

\newcommand{\R}{\mathbb{R}}
\newcommand{\N}{\mathbb{N}}
\newcommand{\Q}{\mathbb{Q}}
\newcommand{\Z}{\mathbb{Z}}
\newcommand{\C}{\mathbb{C}}
\newcommand{\K}{\mathbb{K}}
\newcommand{\eqi}{\Leftrightarrow}
\titleformat{\subsubsection}
   {\normalfont\fontsize{11pt}{13pt}\selectfont\bfseries}% apparence commune au titre et au num�ro
   {\thesubsubsection}% apparence du num�ro
   {1em}% espacement num�ro/texte
   {}% apparence du titre

\tikzstyle{theobox} = [draw=black, very thick,
    rectangle, rounded corners, inner sep=10pt, inner ysep=20pt]
\tikzstyle{theotitle} =[fill=white, text=black,rounded corners,draw=black,very thick]

\fancyhead[L]{Contrôle chapitre 10}

\usepackage[c]{esvect}
\newcommand{\covec}[2]{\begin{pmatrix}#1 \\#2 \end{pmatrix}}

%\usepackage[latin1]{inputenc}
%\usepackage[T1]{fontenc}
%\usepackage[colorlinks=true,urlcolor=black]{hyperref}
\usepackage{fancyhdr}
%FONTES POUR ALGO
\DeclareFontFamily{T1}{lmtt}{} 
\DeclareFontShape{T1}{lmtt}{m}{n}{<-> ec-lmtl10}{} 
\DeclareFontShape{T1}{lmtt}{m}{\itdefault}{<-> ec-lmtlo10}{} 
\DeclareFontShape{T1}{lmtt}{\bfdefault}{n}{<-> ec-lmtk10}{} 
\DeclareFontShape{T1}{lmtt}{\bfdefault}{\itdefault}{<-> ec-lmtko10}{}
\renewcommand{\ttdefault}{lmtt}
% PACKAGES NECESSAIRES POUR ALGO
\usepackage{xcolor}
\usepackage{framed}
\usepackage{algorithm}
\usepackage{algpseudocode}
%ALGO
\definecolor{fond}{RGB}{136,136,136}
\definecolor{sicolor}{RGB}{128,0,128}
\definecolor{tantquecolor}{RGB}{221,111,6}
\definecolor{pourcolor}{RGB}{187,136,0}
\definecolor{bloccolor}{RGB}{128,0,0}
\newenvironment{cadrecode}{%
\def\FrameCommand{{\color{fond}\vrule width 5pt}\fcolorbox{fond}{white}}%
\MakeFramed {\hsize \linewidth \advance\hsize-\width \FrameRestore}\begin{footnotesize}}%
{\end{footnotesize}\endMakeFramed}
\makeatletter
\def\therule{\makebox[\algorithmicindent][l]{\hspace*{.5em}\color{fond} \vrule width 1pt height .75\baselineskip depth .25\baselineskip}}%
\newtoks\therules
\therules={}
\def\appendto#1#2{\expandafter#1\expandafter{\the#1#2}}
\def\gobblefirst#1{#1\expandafter\expandafter\expandafter{\expandafter\@gobble\the#1}}%
\def\Ligne{\State\unskip\the\therules}% 
\def\pushindent{\appendto\therules\therule}%
\def\popindent{\gobblefirst\therules}%
\def\printindent{\unskip\the\therules}%
\def\printandpush{\printindent\pushindent}%
\def\popandprint{\popindent\printindent}%
\def\Variables{\Ligne \textcolor{bloccolor}{\textbf{VARIABLES}}}
\def\Si#1{\Ligne \textcolor{sicolor}{\textbf{SI}} #1 \textcolor{sicolor}{\textbf{ALORS}}}%
\def\Sinon{\Ligne \textcolor{sicolor}{\textbf{SINON}}}%
\def\Pour#1#2#3{\Ligne \textcolor{pourcolor}{\textbf{POUR}} #1 \textcolor{pourcolor}{\textbf{ALLANT\_DE}} #2 \textcolor{pourcolor}{\textbf{A}} #3}%
\def\Tantque#1{\Ligne \textcolor{tantquecolor}{\textbf{TANT\_QUE}} #1 \textcolor{tantquecolor}{\textbf{FAIRE}}}%
\algdef{SE}[WHILE]{DebutTantQue}{FinTantQue}
  {\pushindent \printindent  \textcolor{tantquecolor}{\textbf{DEBUT\_TANT\_QUE}}}
  {\printindent \popindent  \textcolor{tantquecolor}{\textbf{FIN\_TANT\_QUE}}}%
\algdef{SE}[FOR]{DebutPour}{FinPour}
  {\pushindent \printindent \textcolor{pourcolor}{\textbf{DEBUT\_POUR}}}
  {\printindent \popindent  \textcolor{pourcolor}{\textbf{FIN\_POUR}}}%
\algdef{SE}[IF]{DebutSi}{FinSi}%
  {\pushindent \printindent \textcolor{sicolor}{\textbf{DEBUT\_SI}}}
  {\printindent \popindent \textcolor{sicolor}{\textbf{FIN\_SI}}}%
\algdef{SE}[IF]{DebutSinon}{FinSinon}
  {\pushindent \printindent \textcolor{sicolor}{\textbf{DEBUT\_SINON}}}
  {\printindent \popindent \textcolor{sicolor}{\textbf{FIN\_SINON}}}%
\algdef{SE}[PROCEDURE]{DebutAlgo}{FinAlgo}
   {\printandpush \textcolor{bloccolor}{\textbf{DEBUT\_ALGORITHME}}}%
   {\popandprint \textcolor{bloccolor}{\textbf{FIN\_ALGORITHME}}}%
\makeatother
\newenvironment{algobox}%
{%
\begin{ttfamily}
\begin{algorithmic}[1]
\begin{cadrecode}
\labelwidth 1.5em
\leftmargin\labelwidth \addtolength{\leftmargin}{\labelsep}
}
{%
\end{cadrecode}
\end{algorithmic}
\end{ttfamily}
}



\begin{document}
\center
\Large Contrôle de cours (correction)
\flushleft
\center
Variables aléatoires
\flushleft \normalsize
\subsection*{Exercice 1 (R.O.C., temps conseillé : 10 min) : }
Soit $X$ une variable aléatoire sur $(\Omega,P)$. On note $X(\Omega) = \{x_1,x_2,\ldots,x_p\}$. 
On définit l'espérance par 
$$E(X) = x_1P(X=x_1) + x_2P(X=x_2) + \ldots + x_pP(X=x_p) = \sum_{i=1}^p x_i P(X=x_i)$$
La variance est définie par $$V(X) = E((X-E(X))^2)$$ c'est à dire, en note $\mu = E(X)$
$$V(X) = (x_1-\mu)^2 P(X=x_1) + (x_2-\mu)^2 P(X=x_2) + \ldots + (x_p-\mu)^2 P(X=x_p)$$ 

Finalement l'écart type de $X$ est donné par 
$$\sigma(X) = \sqrt{V(X)}$$ (qui existe bien car $V(X)\geq0$). 
On note $Y=aX+b$, alors 
$$E(Y) =  \sum_{i=1}^p y_i P(Y=y_i)$$
où pour tout $i$, $y_i =ax_i+b$
$$E(Y) =  \sum_{i=1}^p (ax_i+b) P(aX+b=ax_i+b)$$
$$E(Y) =  \sum_{i=1}^p (ax_i+b) P(X=x_i)$$
$$E(Y) =  \sum_{i=1}^p ax_iP(X=x_i) + \sum_{i=1}^p bP(X=x_i)$$
$$E(Y) =  a\sum_{i=1}^p x_iP(X=x_i) + b\sum_{i=1}^p P(X=x_i)$$
or $\displaystyle \sum_{i=1}^p x_iP(X=x_i) = E(X)$ et $\displaystyle \sum_{i=1}^p P(X=x_i) = 1$ donc 
$$\boxed{E(aX+b) = aE(X) +b}$$
De plus on a 
$$V(aX) = E((aX-E(aX))^2) = E(a^2(X-E(X))^2) = a^2E((X-E(X))^2) = a^2 V(X)$$
\subsection*{Exercice 2 (Un jeu de casino, temps conseillé : 20 min) : }
Gros attention : la mise de départ doit être comptabilisé dans les gains : quand "le joueur ne gagne rien", il perd en fait sa mise de départ soit 2 euros
\begin{enumerate}
\item L'idée la plus simple ici est de faire un diagramme de Venn et d'en déduire les probabilités\newline
Il a une seule carte qui a la bonne couleur et la bonne valeur soit 
$$\boxed{P(A) = \frac{1}{52}}$$
Il a 13 cartes qui ont la couleur que le joueur a donné, et 12 qui ont la même couleur sans avoir la bonne valeur soit 
$$\boxed{P(B) = \frac{12}{52}}$$
Il y 4 cartes qui ont la même valeur que celle donnée par le joueur dont 3 qui n'ont pas la même couleur donc 
$$\boxed{P(C) = \frac{3}{52}}$$
Finalement, il a $52-1-12-3 = 36$ cartes qui n'ont rien en commun donc 
$$\boxed{P(D) = \frac{36}{52}}$$
\item $$\boxed{X(\Omega) =\{-2,0,20\}}$$
\item La probabilité d'obtenir 20 euros est la même que celle d'obtenir la bonne carte soit $$\frac{1}{52}$$ La probabilité de ne rien gagner est celle d'avoir la bonne couleur uniquement additionnée à celle d'avoir la bonne valeur uniquement soit 
$$\frac{12}{52} + \frac{3}{52} = \frac{15}{52}$$
Finalement la probabilité de perdre 2 euros est $$\frac{36}{52}$$ Ainsi, on peut donner la loi de $X$ dans un tableau \newline
\begin{tabularx}{\linewidth}{|X|X|X|X|}
\hline
$x$ & $-2$ & $0$ & $20$\\ \hline
$P(X=x)$ & $ \frac{36}{52}$ & $ \frac{15}{52}$ & $ \frac{1}{52}$ \rule[-7pt]{0pt}{20pt} \\ \hline
\end{tabularx}
\item On a $$E(X) = -2 \times \frac{36}{52} + 0 \times \frac{15}{52} + 20 \times \frac{1}{52}$$ $$\boxed{E(X) = -1}$$ Sur un grand nombre de partie, on aura perdu, en moyenne, 1 euros par partie don non très peu pour moi.
\item En moyenne il aura perdu 1 euro par partie soit une perte de 100 euros
\item On a $$V(X) = (-2-(-1))^2\times \frac{36}{52} + (0 -(-1))^2 \times \frac{15}{52} + (20 -(-1))^2 \times \frac{1}{52} = \frac{36}{52} + \frac{15}{52} + 21^2 \times \frac{1}{52}$$
$$\boxed{V(X) = \frac{123}{13} \simeq 9,46}$$

Un grand écart-type qui se traduit pa run éloignement des gains extrêmes.
\item En considérant "avoir la bonne combinaison" comme un succès, on trouve que $Y$ suit la loi binomiale de paramètre $n$ et $\displaystyle \frac{1}{52}$ : $$\boxed{Y \sim \mathscr{B}\left(n,\frac{1}{52}\right)}$$
\item On a 
$$P(Y \leq 2) = P(Y=0) + P(Y=1) + P(Y=2)$$
On utilisant le fait que 
$$P(X=k) = \binom{50}{k} \frac{1}{52}^k \frac{51}{52}^{50-k}$$
(et en demandant gentiment à la calculatrice de fournir la réponse), 
$$\boxed{P(Y \leq 2) = 0,928}$$
Cela veut aussi dire que $P(Y> 2) =0,072$ ce qui veut dire que la probabilité d'obtenir 3 succès ou plus su 50 essais est extrêmement faible...
\item La formule du cours donne 
$$\boxed{E(Y) = \frac{n}{52}}$$ 
On veut $$E(Y) \geq 2$$ ($E(Y) =2$ si on veut) 
$$E(Y) \geq 2$$
$$\frac{n}{52} \geq 2$$
$$n \geq 104$$
Pour $n = 104$, $E(Y) = 2$. Ainsi, à partir de $n=104$, le joueur peut espérer avoir 2 succès. En calculant la variance, on trouve une variance d'environ $1,96$ pour $n=104$ donc des valeurs pas très éloignés de $2$.
\end{enumerate}
\subsection*{Exercice 3 (Un peu de loi binomiale, temps conseillé : 15 min) : }
\begin{enumerate}
\item On dit que $X$ suit la loi binomiale de paramètre $n$ et $p$ si $$X(\Omega) = [\![0,n]\!]$$ et pour tout $k \in [\![0,n]\!]$, $$P(X = k) = \binom{n}{k} p^k(1-p)^{n-k}$$ Si $X \sim \mathscr{B}(n,p)$ alors $$E(X) = np$$ et $$V(X) = np(1-p)$$ 
\item On a $$\boxed{X\sim \mathscr{B}(20,0.7)}$$
\item On calcule $$P(X = 13) = \binom{20}{13}\times0.7^{13}\times0.3^7$$
$$\boxed{P(X=13) = 0,164262}$$
\item On calcule 
$$\sum_{k=0}^7 P(X=k)$$ et en demandant gentiment à la calculatrice de nous fournir la réponse 
$$\boxed{P(X=7) = 0,001279}$$
\item On calcule l'espérance de $X$ : $$E(X) = 20 \times 0,7$$
$$\boxed{E(X) = 14}$$
En moyenne, il mettra 14 fléchettes par soir
\item On a 
$$V(X) = 20\times 0.7 \times 0.3$$
$$\boxed{V(X) = 4.2}$$
\item Algorithme de la factorielle
\begin{algobox}
\Variables
\Ligne n EST\_DU\_TYPE NOMBRE
\Ligne res EST\_DU\_TYPE NOMBRE
\Ligne k EST\_DU\_TYPE NOMBRE
\DebutAlgo
\Ligne SAISIR n
\Ligne res PREND\_LA\_VALEUR 1
\Pour{k}{2}{n}
\DebutPour
\Ligne res PREND\_LA\_VALEUR res * k
\FinPour
\Ligne AFFICHER res
\FinAlgo

\end{algobox}
\item Algorithme des nombres binomiaux
\begin{algobox}
\Variables
\Ligne n EST\_DU\_TYPE NOMBRE
\Ligne k EST\_DU\_TYPE NOMBRE
\Ligne res EST\_DU\_TYPE NOMBRE
\DebutAlgo
\Ligne SAISIR n
\Ligne SAISIR k
\Ligne res PREND\_LA\_VALEUR factorielle(n)/(factorielle(k)*factorielle(n-k))
\Ligne AFFICHER res
\FinAlgo

\end{algobox}
\item Algorithme de $P(X=k)$
\begin{algobox}
\Variables
\Ligne p EST\_DU\_TYPE NOMBRE
\Ligne n EST\_DU\_TYPE NOMBRE
\Ligne k EST\_DU\_TYPE NOMBRE
\Ligne res EST\_DU\_TYPE NOMBRE
\DebutAlgo
\Ligne SAISIR n
\Ligne SAISIR p
\Ligne SAISIR k
\Ligne res PREND\_LA\_VALEUR binom(k,n)*pow(p,k)*pow(1-p,n-k)
\Ligne AFFICHER res
\FinAlgo
\end{algobox}
\end{enumerate}
\subsection*{Exercice 4 (Loi géométrique, temps conseillé : 30-40 min) : }
\begin{enumerate}
\item La flemme de faire l'arbre, mais en gros pour que le premier succès arrive en $k$, il faut que les $(k-1)$ premières branches soient $E$ (échec) et la dernière $S$ (succès) soit une probabilité
$$\underbrace{(1-p)\times (1-p) \times \ldots \times (1-p)}_{n-1 \text{ fois}} \times p = (1-p)^{n-1} p$$
\item On a
$$\boxed{\sum_{k=0}^n q^k = 1 + q + q^2 + \ldots + q^n = \frac{1-q^{n+1}}{1-q}}$$
où $q$ est un réel différent de $1$
\item Comparer, avec des petits points 
$$\sum_{k=0}^{n-1} a_k = a_0 + a_1 + \ldots + a_{n-1}$$ et $$\sum_{k=1}^n a_{k-1} = a_{1-1} + a_{2-1} + \ldots + a_0 + a_1 + \ldots + a_{n-1}$$
Ainsi 
$$\boxed{\sum_{k=0}^{n-1} a_k = \sum_{k=1}^n a_{k-1}}$$
Ainsi 
$$\sum_{k=1}^n q^{k-1} = \sum_{k=0}^{n-1} q^k$$
\item On a
$$\sum_{k=1}^n P(X=k) = \sum_{k=1}^n (1-p)^{k-1} p = p\sum_{k=1}^n (1-p)^{k-1}$$ or d'après la question précédente et celle d'avant $$\sum_{k=1}^n (1-p)^{k-1}=\sum_{k=0}^{n-1} (1-p)^{k} = \frac{1-(1-p)^n}{1-(1-p)} = \frac{1-(1-p)^2}{p}$$

Ainsi 
$$\sum_{k=1}^n P(X=k) = p\sum_{k=1}^n (1-p)^{k-1} = p\frac{1-(1-p)^n}{p}$$
Finalement 
$$\boxed{\sum_{k=1}^n P(X=k) = 1 - (1-p)^n}$$
\item On a $0 < 1-p < 1$ car $0<p<1$ donc $$(1-p)^n \underset{n\rightarrow \infty}{\longrightarrow} 0$$ donc $$\underset{n\rightarrow \infty}{\lim} \sum_{k=1}^n P(X=k) = 1 - \underset{n\rightarrow \infty}{\lim} (1-p)^n$$ d'où $$\boxed{\sum_{k=1}^\infty P(X=k)=1}$$ Oui c'est rassurant
\item On veut ici établir le résultat suivant : 
$$\boxed{\sum_{k=1}^\infty kx^{k-1} = \frac{1}{(1-x)^2}}$$ pour tout $x\in ]0,1[$
\begin{enumerate}
\item On note $$f_n(x) = \sum_{k=0}^n x^k$$ $f_n$ est dérivable comme somme de fonctions dérivables et pour tout $x\in ]0,1[$, $$\boxed{f'_n(x) = \sum_{k=0} kx^{k-1} = \sum_{k=1} kx^{k-1}}$$ 
\item On a $$f_n(x) = \sum_{k=0}^n x^k = \frac{1-x^{n+1}}{1-x}$$ $f_n$ est dérivable comme quotient d'une fonction dérivable et d'une fonction dérivable qui ne s'annule pas. On a de plus 
$$\boxed{f'_n(x) = \frac{-(n+1)x^n(1-x) +(1-x^{n+1})}{(1-x)^2}}$$
\item On déduit des deux questions précédentes
$$\boxed{\sum_{k=1}^n kx^{k-1} = \frac{-(n+1)x^n(1-x) +(1-x^{n+1})}{(1-x)^2}}$$
\item On a $$x^n \underset{n\rightarrow \infty}{\longrightarrow} 0$$ et $$x^{n+1} \underset{n\rightarrow \infty}{\longrightarrow} 0$$ donc $$-(n+1)x^n(1-x) +(1-x^{n+1}) \underset{n\rightarrow \infty}{\longrightarrow} 1$$
Finalement
$$\boxed{\sum_{k=1}^\infty kx^{k-1} = \frac{1}{(1-x)^2}}$$
\end{enumerate}
\item On suppose $X\sim \mathcal{G}(p)$, alors 
$$E(X) = \sum_{k=1}^\infty kP(X=k)$$
$$E(X) = \sum_{k=1}^\infty k(1-p)^{k-1}p$$
$$E(X) = p\sum_{k=1}^\infty k(1-p)^{k-1}$$
$$E(X) = p\frac{1}{(1-(1-p))^2}$$
$$E(X) = p\frac{1}{p^2}$$
$$\boxed{E(X) = \frac{1}{p}}$$
\item On suppose $X\sim \mathcal{G}(p)$
$$V(X) = E(X^2)-E(X)^2$$
$$V(X) = \sum_{k=1}^\infty k^2 (1-p)^{k-1}p - \frac{1}{p^2}$$
$$V(X) = p\sum_{k=1}^\infty k^2 (1-p)^{k-1} - \frac{1}{p^2}$$
$$V(X) = p\left( \frac{2}{(1-(1-p))^3}-\frac{1}{(1-(1-p))^2}\right) - \frac{1}{p^2}$$
$$V(X) = p\left( \frac{2}{p^3}-\frac{1}{p^2}\right) - \frac{1}{p^2}$$
$$V(X) = \frac{2}{p^2}-\frac{1}{p} - \frac{1}{p^2}$$
$$V(X) = \frac{1}{p^2}-\frac{1}{p}$$
$$V(X) = \frac{1}{p^2}-\frac{p}{p^2}$$
$$\boxed{V(X) = \frac{1-p}{p^2}}$$
\item 
\begin{enumerate} 
\item La probabilité de tirer l'as de coeur $\frac{1}{4}$ donc la loi qui donne le nombre de tirage est celle du premier succès sur une expérience où le succès est à $\frac{1}{4}$.
\item On a $$\boxed{P(X = 3) = \left(\frac{3}{4}\right)^2\times \frac{1}{4} = \frac{9}{64}}$$ 
\item On a $$P(X\geq7) = \sum_{k=1}^7 P(X=k)$$ puis en demandant gentiment à la calculatrice fournir la réponse 
$$\boxed{P(X\geq 7) = 0,866516}$$
\item On $$E(X) = \frac{1}{\frac{1}{4}} = 4$$ et $$V(X) = \frac{\frac{3}{4}}{\frac{1}{16}} = 12$$ En moyenne, il faut tirer 4 fois avant d'obtenir l'as de coeur 
\item Un peu mais ça va\end{enumerate}
\end{enumerate}
$$\star \star \star$$
\center
FIN DU SUJET
\end{document}