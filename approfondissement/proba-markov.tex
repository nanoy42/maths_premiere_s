\documentclass{article}[11pt]
\usepackage[frenchb,english]{babel}
\usepackage[T1]{fontenc}
\usepackage[utf8]{inputenc}
\usepackage{amsmath,amssymb,latexsym}
\usepackage{times}
\usepackage{float}
\usepackage[left=2cm,right=2cm,top=2cm,bottom=2cm]{geometry}
\frenchbsetup{StandardLists=true} % � inclure si on utilise \usepackage[french]{babel}
\usepackage{enumitem}
\usepackage{fancyhdr}
\usepackage{mathrsfs}
\usepackage{graphicx}
%\usepackage[Algorithme]{algorithm}
%\usepackage{algorithmic}
\usepackage{tikz}
\usepackage{tabularx}
\usetikzlibrary{shapes}
\pagestyle{fancy}
\newcommand{\tr}[1]{{\vphantom{#1}}^{\mathit t}{#1}} 
\renewcommand\headrulewidth{1pt}
\fancyhead[L]{Cours 1�re S}
\fancyhead[R]{Yoann Pietri}
\newcounter{theoremecounter}[subsection]
\usepackage{titlesec}
\setcounter{secnumdepth}{3}% enl�ve la num�rotation apr�s les sections
%\renewcommand\thechapter {\Roman{chapter}}

 \setlength{\parindent}{0pt}

\newcommand{\R}{\mathbb{R}}
\newcommand{\N}{\mathbb{N}}
\newcommand{\Q}{\mathbb{Q}}
\newcommand{\Z}{\mathbb{Z}}
\newcommand{\C}{\mathbb{C}}
\newcommand{\K}{\mathbb{K}}
\newcommand{\eqi}{\Leftrightarrow}
\titleformat{\subsubsection}
   {\normalfont\fontsize{11pt}{13pt}\selectfont\bfseries}% apparence commune au titre et au num�ro
   {\thesubsubsection}% apparence du num�ro
   {1em}% espacement num�ro/texte
   {}% apparence du titre

\tikzstyle{theobox} = [draw=black, very thick,
    rectangle, rounded corners, inner sep=10pt, inner ysep=20pt]
\tikzstyle{theotitle} =[fill=white, text=black,rounded corners,draw=black,very thick]


\usepackage{empheq}
\usepackage[c]{esvect}
\newcommand{\covec}[2]{\begin{pmatrix}#1 \\#2 \end{pmatrix}}

\fancyhead[L]{Sujet : Inégalité de Markov et de Bienaymé-Tchebychev}


\begin{document}
\center{Sujet : Inégalité de Markov et de Bienaymé-Tchebychev}
\flushleft
Si au cours du sujet, le candidat repère ce qui lui semble être une erreur énoncé, il l'indique sur sa copie et continue sa composition. \newline

Il est demandé au candidat une clarté dans les raisonnements qu'il mettra en place. \newline

L'usage de la calculatrice est interdit.\newline

$$\star$$
\underline{Notations :}\newline
Dans tous le problème $\Omega$ désigne un univers et $P$ une probabilité sur $\Omega$ (sauf dans la partie IV où $\Omega$ et $P$ sont particularisés). Si $X$ est une variable aléatoire sur $(\Omega,P)$, on note $E(X)$ l''espérance de $X$ et $V(X)$ la variance de $X$.\newline

$[\![1,n]\!]$ désigne l'ensemble des nombres entiers compris entre $1$ et $n$ ($1$ et $n$ compris)

$$\star$$
L'objectif de ce problème est de prouver les inégalités suivantes : \newline

Pour $X$ une variable aléatoire sur $(\Omega,P)$ avec $X \geq 0$, on a pour tout $a>0$ : 
$$P(X \geq a) \leq \frac{E(X)}{a}$$
$$P(|X-E(X)| \geq a) \leq \frac{V(X)}{a^2}$$
Le sujet est composé de 4 parties qui peuvent être traitées indépendamment \textbf{en admettant les résultats encadrés}\newline

\section*{I - Quelques résultats sur les sommes}
Soient $a_1,a_2,\ldots,a_n$ des réels. On note 
$$a_1 + a_2 + \ldots + a_n = \sum_{k=1}^n a_k = \sum_{k \in [\![1,n]\!]} a_k$$
Si $A$ est un ensemble, on peut aussi écrire 
$$\sum_{x\in A} x$$ qui est la somme de tous les éléments de l'ensemble $A$
\begin{enumerate}
\item Soit $a_1,a_2,\ldots,a_n,b_1,b_2,\ldots,b_n$ des réels. Montrer que 
$$\sum_{k=1}^n a_k + b_k = \sum_{k=1}^n a_k + \sum_{k=1}^n b_k$$
\item Soit $a_1,a_2,\ldots,a_n$ et $\lambda \in \R$. Montrer que 
$$\sum_{k=1}^n \lambda a_k = \lambda \sum_{k=1}^n a_k $$

On rappelle ici que si $a \leq b$ et $c \leq d$ alors $a+c \leq b+d$. \newline

On en déduit que si pour tout $k \in [\![1,n]\!]$, $a_k \leq b_k$ alors $$\sum_{k=1}^n a_k \leq \sum_{k=1}^n b_k$$
\item En déduire que 
$$\boxed{(\forall k \in [\![1,n]\!], a_k \geq 0 )\Rightarrow \sum_{k=1}^n a_k \geq 0}$$
(si tous les $a_k$ sont positifs alors la somme est positive)
\item Montrer que si pour tout $k \in [\![1,n]\!]$, $a_k \geq 0$ et $\sum_{k=1}^n a_k = 0$, alors pour tout $k\in [\![1,n]\!], a_k = 0$
\item On peut ajouter des conditions à des sommes : par exemple 
$$\sum_{\substack{ k \in [\![1,n]\!] \\ a_k \geq 3}} a_k$$ est la somme de tous les $a_k$ plus grand que 3\newline 

Calculer par exemple $$\sum_{\substack{ k \in [\![1,8]\!] \\ a_k \geq 3}} a_k$$ où les $a_k$ sont définis par \newline

\begin{tabularx}{\linewidth}{|X|X|X|X|X|X|X|X|}
\hline
$a_1$ & $a_2$ & $a_3$ & $a_4$ & $a_5$ & $a_6$ & $a_7$ & $a_8$ \\ \hline
-1 & 2 & 4 & 6 & 7 & 2,5& 4 & 2\\
\hline
\end{tabularx}
\end{enumerate}
\section*{II - Inégalité de Markov}
\begin{enumerate}
\item Soit $X$ une variable aléatoire sur $(\Omega,P)$. On note $X(\Omega) = \{x_1,x_2,\ldots,x_n\}$. Ecrire $E(X)$ sous la forme d'une somme (on utilisera $\sum_{k=1}^n$ plutôt que $\sum_{x\in X(\Omega)}$)
\item On dit que $X \geq 0$ si pour tout $x\in X(\Omega)$, $x\geq 0$.\newline

Montrer que 
$$\boxed{X \geq 0 \Rightarrow E(X) \geq 0}$$
A l'aide d'un contre exemple, montrer que l'on peut avoir $E(X) \geq 0$ sans avoir $X \geq 0$
\item Soit $X \geq 0$ et $a > 0$. Montrer que 
$$\boxed{\frac{E(X)}{a} = \sum_{\substack{k \in [ \![1,n]\!]\\x_k \geq a}} \frac{x_k}{a} P(X = x_k)+ \sum_{\substack{k \in [ \![1,n]\!]\\x_k \leq a}} \frac{x_k}{a} P(X = x_k)}$$
(\emph{On ira étape par étape en explicitant bien les opérations effectuées})
\item Le terme $$\sum_{\substack{k \in [ \![1,n]\!]\\x_k \leq a}} \frac{x_k}{a} P(X = x_k)$$ est-il positif ? négatif ? nul ? En déduire que $$\boxed{\frac{E(X)}{a} \geq \sum_{\substack{k \in [ \![1,n]\!]\\x_k \geq a}} \frac{x_k}{a} P(X = x_k)}$$
\item Prouver que 
$$\sum_{\substack{k \in [ \![1,n]\!]\\x_k \geq a}} \frac{x_k}{a} P(X = x_k) \geq \sum_{\substack{k \in [ \![1,n]\!]\\x_k \geq a}} P(X = x_k)$$
(\emph{On attend ici UN argument mais un vrai, pas des justifications fumeuses})
On déduit ainsi 
$$\boxed{\frac{E(X)}{a} \geq  \sum_{\substack{k \in [ \![1,n]\!]\\x_k \geq a}} P(X = x_k)}$$
\item Ecrire $P(X \geq a)$ sous la forme d'une somme dont les termes sont de la forme $P(X = x_k)$ 
\item En déduire l'inégalité de Markov $$\boxed{P(X \geq a)\leq \frac{E(X)}{a}}$$
\end{enumerate}
\section*{III - Inégalité de Bienaymé-Tchebychev}
Soit $Y$ une variable aléatoire et soit $b \in \R$. On dit que $Y \geq b$ si pour tout $y \in Y(\Omega)$, on a $y \geq b$
\begin{enumerate}
\item Soit $b> 0$. Montrer que 
$$\boxed{Y^2 \geq b^2 \Leftrightarrow |Y| \geq b}$$
En déduire que $$P(Y^2 \geq b^2) = P(|Y| \geq b)$$
\item Rappeler la définition de $V(X)$
\item Montrer que $$V(X) = E(X^2)-E(X)$$
\item Soit $a > 0$. Appliquer l'inégalité de Markov à $(X-E(X))^2$ et $a^2$
\item Déduire de la question précédente et de la question III.1 l'inégalité de Bienaymé-Tchebychev $$\boxed{P(|X - E(X)| \geq a) \leq \frac{V(X)}{a^2}}$$
\end{enumerate}
\section*{IV - 2 applications}
\begin{enumerate}
\item On considère le jeu suivant : \emph{On lance deux dés équilibrés et on gagne, en euros, la somme des deux dés} (par exemple, si on tire un 3 et un 6, on gagne 9 euros). On note $X$ la variable aléatoire qui donne le gain en euros 
\begin{enumerate}
\item Donner $\Omega$, $X(\Omega)$, la loi de $X$, $E(X)$ et $V(X)$. Vérifier aussi que $X \geq 0$
\item Ecrire l'inégalité de Markov pour l'évènement, le gain est plus grand que 8 euros ($P(X \geq 8)$)
\item Calculer $P(X \geq 8)$ et conclure
\end{enumerate}
\item Pourquoi peut-on dire que la variance caractérise l'écart entre les valeurs d'une variable aléatoire $Y$ (c'est à dire l'écart à l'espérance). On pourra utiliser l'inégalité de Bienaymé-Tchebychev
\end{enumerate}
\end{document}