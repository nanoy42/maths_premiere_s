\documentclass{article}[11pt]
\usepackage[frenchb,english]{babel}
\usepackage[T1]{fontenc}
\usepackage[utf8]{inputenc}
\usepackage{amsmath,amssymb,latexsym}
\usepackage{times}
\usepackage{float}
\usepackage[left=2cm,right=2cm,top=2cm,bottom=2cm]{geometry}
\frenchbsetup{StandardLists=true} % � inclure si on utilise \usepackage[french]{babel}
\usepackage{enumitem}
\usepackage{fancyhdr}
\usepackage{mathrsfs}
\usepackage{graphicx}
%\usepackage[Algorithme]{algorithm}
%\usepackage{algorithmic}
\usepackage{tikz}
\usepackage{tabularx}
\usetikzlibrary{shapes}
\pagestyle{fancy}
\newcommand{\tr}[1]{{\vphantom{#1}}^{\mathit t}{#1}} 
\renewcommand\headrulewidth{1pt}
\fancyhead[L]{Cours 1�re S}
\fancyhead[R]{Yoann Pietri}
\newcounter{theoremecounter}[subsection]
\usepackage{titlesec}
\setcounter{secnumdepth}{3}% enl�ve la num�rotation apr�s les sections
%\renewcommand\thechapter {\Roman{chapter}}

 \setlength{\parindent}{0pt}

\newcommand{\R}{\mathbb{R}}
\newcommand{\N}{\mathbb{N}}
\newcommand{\Q}{\mathbb{Q}}
\newcommand{\Z}{\mathbb{Z}}
\newcommand{\C}{\mathbb{C}}
\newcommand{\K}{\mathbb{K}}
\newcommand{\eqi}{\Leftrightarrow}
\titleformat{\subsubsection}
   {\normalfont\fontsize{11pt}{13pt}\selectfont\bfseries}% apparence commune au titre et au num�ro
   {\thesubsubsection}% apparence du num�ro
   {1em}% espacement num�ro/texte
   {}% apparence du titre

\tikzstyle{theobox} = [draw=black, very thick,
    rectangle, rounded corners, inner sep=10pt, inner ysep=20pt]
\tikzstyle{theotitle} =[fill=white, text=black,rounded corners,draw=black,very thick]


\fancyhead[L]{Sujet : Les complexes}

\newcommand{\covec}[2]{\begin{pmatrix}#1 \\#2 \end{pmatrix}}

\begin{document}
\center{Sujets : les complexes}
\flushleft
Si au cours du sujet, le candidat repère ce qui lui semble être une erreur énoncé, il l'indique sur sa copie et continue sa composition. \newline

Il est demandé au candidat une clarté dans les raisonnements qu'il mettra en place. \newline

L'usage de la calculatrice est interdit.\newline

Soit $A$ un ensemble. On définit une loi sur cette ensemble comme une fonction qui à 2 éléments de $A$ associe un élément de $A$ (exemple : $+$ est une loi sur $\R$ car à $a$ et $b$, elle associe $a+b\in\R$)\newline

On considère une loi $\otimes$ sur un ensemble $A$. 
\begin{itemize}
\item Soit $e\in A$. On dit que $e$ est un neutre pour $\otimes$ si pour tout $x\in A$, on a
$$e\otimes x = x \otimes e = x$$
(par exemple 0 est un neutre pour la loi $+$ sur $\R$ car $a+ 0 = 0+a = a$ pour tout $a$ réel)
\item On dit que $\otimes$ est commutative sir pout tout $x$ et $y$ dans $A$, on a 
$$x\otimes y = y\otimes x$$
(exemple, la loi $+$ est commutative car $a+b=b+a$ pour tout $a$ et $b$ réel)
\item On suppose que $\otimes$ admet un élément neutre $e$. Soit $x\in A$. On dit que $A$ est inversible pour $\otimes$ s'il existe $y$ tel que 
$$x\otimes y = y \otimes x = e$$
(exemple : si $a\in \R$, $a$ est inversible pour la loi $+$ car $a+ (-a) = 0$)
\end{itemize}
Si $\oplus$ et $\otimes$ sont deux lois sur $\R$ on dit que $\otimes$ est distributive sur $\oplus$ si pour tout $x,y,z\in A$, on a 
$$x\otimes(y\oplus z) = (x\otimes y) \oplus (x\otimes z)$$
\section{Préliminaires}
\begin{enumerate}
\item Soit $A$ un ensemble et $\otimes$ une loi sur cet ensemble. Soit $e\in A$ et $e' \in A$ deux neutres pour $\otimes$. En calculant $e\otimes e'$ et $e' \otimes e$ ainsi qu'en utilisant les définition d'un neutre, montrer que $$e = e'$$

Le neutre, s'il existe, est unique
\item Soit $A$ un ensemble et $\otimes$ une loi de neutre $e$. Soit $x\in A$ inversible. Soit $y$ et $z$ deux inverses de $x$. En vous inspirant de la question précédente, et en utilisant la définition d'un inverse, montrer $$y = z$$

Un élément inversible admet un unique inverse
\end{enumerate}
\section{Propriétés de $\R^2$}
On note $\R^2$ l'ensemble des couples de réels (c'est-à-dire les points du plan...) 
$$\R^2 = \{(x,y)\}_{x,y\in\R}$$
((1,-3), (2,4) et (3,0) sont par exemples des éléments de $\R^2$\newline

On munit $\R^2$ des lois suivantes : pour tout couples $(x,y)$ et $(x',y')$ de $\R^2$
$$(x,y) + (x',y') = (x+x',y+y')$$
$$\lambda (x,y) = (\lambda x, \lambda y) \quad (\lambda \in \R)$$
$$(x,y) \times (x',y') = (xx' - yy',xy'+x'y)$$
De plus, pour un couple $(x,y)$ de $\R^2$, on note $(x,y)^2 = (x,y) \times (x,y)$\newline

On confond un élément de $\R^2$ de la forme $(x,0)$ avec le nombre réel associé $x$ (en fait on assimile la droite des réels à la droite des abscisses...)\newline

On admet que $$(x,y) = (x',y') \Leftrightarrow \left\{ \begin{array}{l} x = x' \\ y = y' \end{array} \right.$$
\begin{enumerate}
\item Soit $(x,y)$ un couple de $\R^2$. Calculer $(0,0) + (x,y)$
En déduire que $(0,0)$ est un neutre pour $+$ sur $\R^2$
\item Soit $(x,y)$ un couple de $\R^2$. Calculer $(1,0) \times (x,y)$
En déduire que $(1,0)$ est un neutre pour $\times$ sur $\R^2$
\item Soit $(x,y)$ un couple de $\R^2$. montrer que son inverse pour la loi $+$ est le couple $(-x,-y)$
\item Soit $(x,y)$ un couple de $\R^2$. Calculer $(0,0) \times (x,y)$
En déduire que $(0,0)$ n'a pas d'inverse pour la loi $\times$
\item Soit $(x,y)\in \R^2$, $(x,y) \neq (0,0)$. Montrer alors que $(x,y)$ admet un inverse pour la loi $\times$ que l'on note $\frac{1}{(x,y)}$ (on résoudra un système et on explicitera $\frac{1}{(x,y)}$ par ses coordonnées)\newline

Ainsi, pour tout $(x,y) \neq (0,0)$
$$\boxed{(x,y) \times \frac{1}{(x,y)} = (1,0)}$$
\item Montrer $(0,1)^2 = (-1,0)$\newline

Alors en notant $i = (0,1)$ et en confondant $(-1,0)$ avec $-1$ on obtient la relation fondamentale des complexes 
$$\boxed{i^2 = -1}$$
NB : Il n'y a pas d'erreurs d'énoncé : c'est tout le propre des complexes d'avoir des nombres qui, élevés au carré, font des nombres négatifs
\item R.O.C., rappelez le théorème de décomposition selon 2 vecteurs non colinéaires
\item Montrer que les vecteurs $\covec{1}{0}$ et $\covec{0}{1}$ ne sont pas colinéaires
\item Soit $(x,y)\in \R^2$. Ecrire sa décomposition selon $(1,0)$ et $(0,1)$.\newline

On déduit que tout élément $z$ de $\R^2$, on peut trouver $a$ et $b$ tels que 
$$z = a(1,0) + b(0,1) = a + ib$$
Cette forme est appelé écriture algébrique.\newline

A partir de maintenant 
$$\C = \{a+ib\}_{a,b\in\R}$$
On a $\C \simeq \R^2$ (c'est presque égal, il y a juste une ou deux petites nuances)\newline

Si $z\in \C$, on dit que $z$ est un complexe. $\C$ est appelé l'ensemble des complexes et on a $\R \subset \C$\newline

Si $z = a+ib$, $a$ est appelée partie réelle de $z$ et notée $Re(z)$ et $b$ est appelée partie imaginaire de $z$ et notée $Im(z)$\newline

Ainsi pour tout $z\in\C$ $$\boxed{z = Re(z) + i Im(z)}$$

Si $Im(z) = 0$, alors $z$ est réel\newline
Si $Re(z) = 0$, on dit que $z$ est un imaginaire pur
Si $Im(z) = Re(z) = 0$, alors $z = 0$\newline

On admet que $$a+ib = a'+ib' \Leftrightarrow \left\{ \begin{array}{l} a = a' \\ b = b' \end{array} \right.$$
(identification des parties réelles et imaginaires)\newline

\item On admet que la loi $\times$ est distributive sur $+$. On considère le nombre complexe $3+2i$. Donner sa partie réelle, sa partie imaginaire et calculer $(3+2i)^2$
\item Soit $z\in\C$. Montrer, en utilisant la définition ou la distributivité de $\times$ sur $+$, que  
$$\boxed{z^2 = Re(z)^2 - Im(z)^2 + 2iRe(z)Im(z)}$$
\end{enumerate}
\section{Conjugué}
Soit $z = a+ib \in \C$. On appelle conjugué de $z$ et on note $\overline{z}$ le nombre complexe $$\overline{z} = a - ib$$
\begin{enumerate}
\item Calculer $\overline{3 + i}$
\item Soit $z\in\C$ \begin{enumerate} \item On suppose $Im(z) = 0$. Montrer que $\overline{z} = z$ \item On suppose $\overline{z} = z$. Montrer que $Im(z) = 0$\end{enumerate} Ainsi on a montré $$\boxed{\overline{z} = z \Leftrightarrow Im(z) = 0 \Leftrightarrow z \text{ réél}}$$
\item Montrer que $$\overline{\overline{z}} = z$$
\item Exprimer, en fonction de $Re(z)$ et $Im(z)$, le nombre 
$$z \times \overline{z}$$
En déduire que $z \times \overline{z}\in\R$
\item Soit $z$ et $z'$ complexes $$\overline{z + z'} = \overline{z} + \overline{z'}$$
\item Soit $z$ et $z'$ complexes $$\overline{z \times z'} = \overline{z} \times \overline{z'}$$
\item Soit $z\neq 0$. Montrer que 
$$\overline{\frac{1}{z}} = \frac{1}{\overline{z}}$$
\end{enumerate}
\section{Module}
Soit $z\in\C$. On appelle module de $z$ et on note $$|z| = \sqrt{Re(z)^2 + Im(z)^2}$$
\begin{enumerate}
\item Calculer le module de $-4+3i$
\item Montrer que 
$$|z|^2 = z\times \overline{z}$$
\item Montrer l'équivalence 
$$|z| = 0 \Leftrightarrow z = 0$$
(on raisonnera par double équivalence en montrant $(z = 0) \Rightarrow (|z| = 0)$ puis $(|z| = 0) \Rightarrow (z = 0)$
\item Soit $z$ et $z'$ deux nombres complexes. Montrer que $$|z\times z' = |z|\times |z'|$$
\item Soit $z\in\C$ et soit $n\in \N^*$ Montrer que 
$$|z^n| = |z|^n$$
\item Soit $z \neq 0$. Montrer que 
$$\left|\frac{1}{z}\right| = \frac{1}{|z|}$$
En déduire que si $z$ et $z'$ sont deux nombres complexes avec $z'\neq 0$, on a 
$$\left|\frac{z}{z'}\right| = \frac{z}{|z'|}$$
\item En vous inspirant d'une démonstration vue en cours, montrer que 
$$|z+z'| \leq |z| + |z'|$$
\end{enumerate}
\section{Retour sur les trinômes du second degré}
\begin{enumerate}
\item Montrer que $i$ est solution de l'équation 
$$x^2+1 = 0$$
Calculer le discriminant du trinôme $x\mapsto x^2 +1$. Que remarquez vous ?
\item Soit $a$ réel avec $a<0$. Montrer que 
$$i\sqrt{-a}$$
est d'une part bien défini et d'autre part solution de 
$$x^2 = a$$
\item R.O.C. Rappelez, sans le démontrer, la forme canonique d'un trinôme $f:ax^2+bx+c$ ($a\neq0$) en fonction de ses coefficients et de son discriminant. Vous démontrerez ensuite le résultat sur les racines d'un trinôme lorsque $\Delta > 0$ et lorsque $\Delta = 0$
\item On s'intéresse au cas $\Delta < 0$. Montrer que, dans ce cas, le trinôme $f$ admet 2 racines complexes $x_1$ et $x_2$ que l'on explicitera en fonction des coefficients. On montrera de plus que $$\overline{x_1} = x_2$$
\item Trouver les racines (éventuellement complexes) du trinôme $f:x\mapsto 2x^2 -6x +13$
\end{enumerate}
\section{Racine carrée d'un nombre complexe}
\begin{enumerate}
\item Soit $z_1 = a+ib$. On s'intéresse aux solutions de l'équation $$z^2 = a+ib$$. On note $z = x+iy$. En déduire un système de deux équations à deux inconnues, puis que pour tout nombre complexe $z_1$, l'équation $z^2 = a+ib$ a des solutions. On donnera notamment des relations entre $x$,$y$,$a$ et $b$
\item Trouver une solution de l'équation 
$$z^2 = 7 + 2i$$
\end{enumerate}
\end{document}