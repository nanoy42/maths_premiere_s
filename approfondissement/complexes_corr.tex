\documentclass{article}[11pt]
\usepackage[frenchb,english]{babel}
\usepackage[T1]{fontenc}
\usepackage[utf8]{inputenc}
\usepackage{amsmath,amssymb,latexsym}
\usepackage{times}
\usepackage{float}
\usepackage[left=2cm,right=2cm,top=2cm,bottom=2cm]{geometry}
\frenchbsetup{StandardLists=true} % � inclure si on utilise \usepackage[french]{babel}
\usepackage{enumitem}
\usepackage{fancyhdr}
\usepackage{mathrsfs}
\usepackage{graphicx}
%\usepackage[Algorithme]{algorithm}
%\usepackage{algorithmic}
\usepackage{tikz}
\usepackage{tabularx}
\usetikzlibrary{shapes}
\pagestyle{fancy}
\newcommand{\tr}[1]{{\vphantom{#1}}^{\mathit t}{#1}} 
\renewcommand\headrulewidth{1pt}
\fancyhead[L]{Cours 1�re S}
\fancyhead[R]{Yoann Pietri}
\newcounter{theoremecounter}[subsection]
\usepackage{titlesec}
\setcounter{secnumdepth}{3}% enl�ve la num�rotation apr�s les sections
%\renewcommand\thechapter {\Roman{chapter}}

 \setlength{\parindent}{0pt}

\newcommand{\R}{\mathbb{R}}
\newcommand{\N}{\mathbb{N}}
\newcommand{\Q}{\mathbb{Q}}
\newcommand{\Z}{\mathbb{Z}}
\newcommand{\C}{\mathbb{C}}
\newcommand{\K}{\mathbb{K}}
\newcommand{\eqi}{\Leftrightarrow}
\titleformat{\subsubsection}
   {\normalfont\fontsize{11pt}{13pt}\selectfont\bfseries}% apparence commune au titre et au num�ro
   {\thesubsubsection}% apparence du num�ro
   {1em}% espacement num�ro/texte
   {}% apparence du titre

\tikzstyle{theobox} = [draw=black, very thick,
    rectangle, rounded corners, inner sep=10pt, inner ysep=20pt]
\tikzstyle{theotitle} =[fill=white, text=black,rounded corners,draw=black,very thick]


\fancyhead[L]{Sujet : Les complexes}

\newcommand{\covec}[2]{\begin{pmatrix}#1 \\#2 \end{pmatrix}}

\begin{document}
\center{Sujets : les complexes}
\flushleft
Si au cours du sujet, le candidat repère ce qui lui semble être une erreur énoncé, il l'indique sur sa copie et continue sa composition. \newline

Il est demandé au candidat une clarté dans les raisonnements qu'il mettra en place. \newline

L'usage de la calculatrice est interdit.\newline
\section{Préliminaires}
\begin{enumerate}
\item Par définition, on a $$e \otimes e' = e'\otimes e = e'$$ or de même, par définition $$e'\otimes e = e$$ donc $$\boxed{e = e'}$$
\item On a $$x\otimes y =e$$ et $$x\otimes z = e$$ donc $$x\otimes y = x\otimes z$$ donc $$y\otimes(x\otimes y) = y\otimes(x\otimes z)$$ donc $$e\otimes y = e\otimes z$$ donc $$\boxed{y = z}$$
\end{enumerate}
\section{Propriétés de $\R^2$}
\begin{enumerate}
\item On a $$(x,y) + (0,0) = (x+0,y+0) = (x,y)$$
Pour tout $(x,y)\in\R^2$, $(x,y) + (0,0) = (x,y)$ donc $(0,0)$ est un neutre pour $+$
\item $$(x,y) \times (1,0) = (x\times 1 - y\times 0,x\times 0 + y\times 1) = (x,y)$$
Pour tout $(x,y)\in\R^2$, $(x,y) \times (1,0) = (x,y)$ donc $(1,0)$ est un neutre pour $\times$
\item On a $$(x,y) + (-x,-y) = (x-x,y-y) = (0,0)$$
\item On a $$(x,y) \times (0,0) = (x\times 0 - y \times 0,y\times 0 + x \times 0) = (0,0)$$
Vu que pour tout $(x,y)\in\R^2$, on a $(x,y)\times(0,0) = (0,0)$ on ne peut pas trouver $(a,b)$ tel que $(0,0)\times (a,b) = (1,0)$ donc $(0,0)$ n'est pas inversible pour $\times$
\item Soit $(x,y)\in \R^2$, $(x,y) \neq (0,0)$. On essaie de résoudre l'équation en z,t
$$(x,y) \times (z,t) = (1,0)$$
$$(xz-ty,xt+yz) = (1,0)$$
$$\left\{ \begin{array}{l}xz-ty = 1\\ xt+yz = 0 \end{array} \right.$$
$$\left\{ \begin{array}{l}x^2z+y^2z = x\\ t = - \frac{yz}{x} \end{array} \right.$$
$$\left\{ \begin{array}{l}z = \frac{x}{x^2+y^2}\\ t = - \frac{yz}{x} \end{array} \right.$$
$$\left\{ \begin{array}{l}z = \frac{x}{x^2+y^2}\\ t = - \frac{y}{x^2+y^2} \end{array} \right.$$
Donc $(x,y)$ est inversible d'inverse $$\boxed{(\frac{x}{x^2+y^2},- \frac{y}{x^2+y^2})}$$
\item $$(0,1)^2 = (0\times 0 - 1\times 1, 0\times 1 + 1\times 0) =(-1,0)$$
$$\boxed{i^2 = -1}$$
\item Soit $\vec{u}$ et $\vec{v}$ deux vecteurs non colinéaires. Alors pour tout vecteur $\vec{w}$, il existe $\lambda$ et $\mu$ tels que 
$$\vec{w} = \lambda \vec{u} + \mu \vec{v}$$
\item $$1\times 1 - 0\times 0 = 1 \neq 0$$ donc les vecteurs ne sont pas colinéaires
\item $$(x,y) = (x,0) + (0,y) = x(1,0) + y(0,1) = x +iy$$
\item $Re(3+2i) = 3$; $Im(3+2i) = 2$; $$(3+2i)^2 = (3+2i)\times(3+2i) = 9 +12i -4$$
\item $$(x+iy)^2 = (x+iy)\times(x+iy) = x\times(x+iy) + iy\times(x+iy) = x^2 + 2ixy +i^2y^2 = x^2 + 2ixy - y^2$$
On obtient bien l'équation demandée
\end{enumerate}
\section{Conjugué}
Soit $z = a+ib \in \C$. On appelle conjugué de $z$ et on note $\overline{z}$ le nombre complexe $$\overline{z} = a - ib$$
\begin{enumerate}
\item $\overline{3 + i} = 3 - i$
\item Soit $z\in\C$ \begin{enumerate} \item On a $z = Re(z) = Re(z) +0\times i$ donc $\overline{z} = Re(z) - 0 \times i = Re(z) = z$ \item On a $\overline{z}=z$ donc $Re(z) + iIm(z) = Re(z) - iIm(z)$ donc $Im(z) = -Im(z)$ donc $Im(z) = 0$ \end{enumerate}
\item $\overline{\overline{z}} = \overline{\overline{Re(z) + iIm(z)}} = \overline{Re(z) - iIm(z)} = Re(z) + iIm(z)=z$
\item 
$$
\begin{array}{l l}
z \times \overline{z} &= (Re(z) + iIm(z))(Re(z) - iIm(z))\\
& = Re(z)^2 +iRe(z)Im(z) - iRe(z)Im(z) -i^2Im(z)\\
& = Re(z)^2 + Im(z)^2
\end{array}
$$
\item $$
\begin{array}{l l}
\overline{z + z'} &= \overline{Re(z)+iIm(z) + Re(z') + iIm(z')} \\
&= \overline{Re(z) + Re(z') + i(Im(z)+Im(z'))} \\
&= Re(z) + Re(z') - i(Im(z)+Im(z'))\\
&= Re(z) - iIm(z) + Re(z') - iIm(z')\\
&= \overline{z} + \overline{z'}
\end{array}
$$
\item $$
\begin{array}{l l}
\overline{z \times z'} &= \overline{(Re(z)+iIm(z)) \times (Re(z') + iIm(z'))} \\
&= \overline{Re(z)Re(z') - Im(z)Im(z') + i(Re(z)Im(z') + Re(z')Im(z))} \\
&= Re(z)Re(z') - Im(z)Im(z') - i(Re(z)Im(z') + Re(z')Im(z))\\
&= \overline{z} \times \overline{z'}
\end{array}
$$
\item $$z \times \frac{1}{z} = 1$$
donc 
$$\overline{z \times \frac{1}{z} } = 1$$
$$\overline{z} \times \overline{\frac{1}{z}} = 1$$
$$\overline{\frac{1}{z}} = \frac{1}{\overline{z}}$$
($\overline{z} \neq 0$ car $z\neq 0$)
\end{enumerate}
\section{Module}
Soit $z\in\C$. On appelle module de $z$ et on note $$|z| = \sqrt{Re(z)^2 + Im(z)^2}$$
\begin{enumerate}
\item $$|-4+3i| = \sqrt{(-4)^2 + (3)^2} = \sqrt{16+9} = 5$$
\item 
$$|z|^2 = Re(z)^2 + Im(z)^2 = z \times \overline{z}$$
(question 3.4)
\item Montrer l'équivalence 
$$|z| = 0 \Leftrightarrow \sqrt{Re(z)^2+Im(z)^2} = 0$$
$$\Leftrightarrow Re(z)^2 + Im(z)^2 = 0$$
$$\Leftrightarrow \left\{\begin{array}{l}Re(z)^2 = 0 \\ Im(z)^2 = 0 \end{array}\right.$$
$$\Leftrightarrow \left\{\begin{array}{l}Re(z) = 0 \\ Im(z) = 0 \end{array}\right.$$
$$\Leftrightarrow z = 0$$
(on raisonnera par double équivalence en montrant $(z = 0) \Rightarrow (|z| = 0)$ puis $(|z| = 0) \Rightarrow (z = 0)$
\item Soit $z$ et $z'$ deux nombres complexes. Montrer que $$|z\times z' = |z|\times |z'|$$
\item Soit $z\in\C$ et soit $n\in \N^*$ Montrer que 
$$|z^n| = |z|^n$$
\item Soit $z \neq 0$. Montrer que 
$$\left|\frac{1}{z}\right| = \frac{1}{|z|}$$
En déduire que si $z$ et $z'$ sont deux nombres complexes avec $z'\neq 0$, on a 
$$\left|\frac{z}{z'}\right| = \frac{z}{|z'|}$$
\item En vous inspirant d'une démonstration vue en cours, montrer que 
$$|z+z'| \leq |z| + |z'|$$
\end{enumerate}
\section{Retour sur les trinômes du second degré}
\begin{enumerate}
\item Montrer que $i$ est solution de l'équation 
$$x^2+1 = 0$$
Calculer le discriminant du trinôme $x\mapsto x^2 +1$. Que remarquez vous ?
\item Soit $a$ réel avec $a<0$. Montrer que 
$$i\sqrt{-a}$$
est d'une part bien défini et d'autre part solution de 
$$x^2 = a$$
\item R.O.C. Rappelez, sans le démontrer, la forme canonique d'un trinôme $f:ax^2+bx+c$ ($a\neq0$) en fonction de ses coefficients et de son discriminant. Vous démontrerez ensuite le résultat sur les racines d'un trinôme lorsque $\Delta > 0$ et lorsque $\Delta = 0$
\item On s'intéresse au cas $\Delta < 0$. Montrer que, dans ce cas, le trinôme $f$ admet 2 racines complexes $x_1$ et $x_2$ que l'on explicitera en fonction des coefficients. On montrera de plus que $$\overline{x_1} = x_2$$
\item Trouver les racines (éventuellement complexes) du trinôme $f:x\mapsto 2x^2 -6x +13$
\end{enumerate}
\section{Racine carrée d'un nombre complexe}
\begin{enumerate}
\item Soit $z_1 = a+ib$. On s'intéresse aux solutions de l'équation $$z^2 = a+ib$$. On note $z = x+iy$. En déduire un système de deux équations à deux inconnues, puis que pour tout nombre complexe $z_1$, l'équation $z^2 = a+ib$ a des solutions. On donnera notamment des relations entre $x$,$y$,$a$ et $b$
\item Trouver une solution de l'équation 
$$z^2 = 7 + 2i$$
\end{enumerate}
\end{document}