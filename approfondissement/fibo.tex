\documentclass{article}[11pt]
\usepackage[frenchb,english]{babel}
\usepackage[T1]{fontenc}
\usepackage[utf8]{inputenc}
\usepackage{amsmath,amssymb,latexsym}
\usepackage{times}
\usepackage{float}
\usepackage[left=2cm,right=2cm,top=2cm,bottom=2cm]{geometry}
\frenchbsetup{StandardLists=true} % � inclure si on utilise \usepackage[french]{babel}
\usepackage{enumitem}
\usepackage{fancyhdr}
\usepackage{mathrsfs}
\usepackage{graphicx}
%\usepackage[Algorithme]{algorithm}
%\usepackage{algorithmic}
\usepackage{tikz}
\usepackage{tabularx}
\usetikzlibrary{shapes}
\pagestyle{fancy}
\newcommand{\tr}[1]{{\vphantom{#1}}^{\mathit t}{#1}} 
\renewcommand\headrulewidth{1pt}
\fancyhead[L]{Cours 1�re S}
\fancyhead[R]{Yoann Pietri}
\newcounter{theoremecounter}[subsection]
\usepackage{titlesec}
\setcounter{secnumdepth}{3}% enl�ve la num�rotation apr�s les sections
%\renewcommand\thechapter {\Roman{chapter}}

 \setlength{\parindent}{0pt}

\newcommand{\R}{\mathbb{R}}
\newcommand{\N}{\mathbb{N}}
\newcommand{\Q}{\mathbb{Q}}
\newcommand{\Z}{\mathbb{Z}}
\newcommand{\C}{\mathbb{C}}
\newcommand{\K}{\mathbb{K}}
\newcommand{\eqi}{\Leftrightarrow}
\titleformat{\subsubsection}
   {\normalfont\fontsize{11pt}{13pt}\selectfont\bfseries}% apparence commune au titre et au num�ro
   {\thesubsubsection}% apparence du num�ro
   {1em}% espacement num�ro/texte
   {}% apparence du titre

\tikzstyle{theobox} = [draw=black, very thick,
    rectangle, rounded corners, inner sep=10pt, inner ysep=20pt]
\tikzstyle{theotitle} =[fill=white, text=black,rounded corners,draw=black,very thick]


\usepackage{tcolorbox}

\fancyhead[L]{Sujet : La suite de Fibonacci}

\begin{document}
\center{\Large{Sujet : la suite de Fibonacci}}
\flushleft
Si au cours du sujet, le candidat repère ce qui lui semble être une erreur d'énoncé, il l'indique sur sa copie et continue sa composition. \newline

Il est demandé au candidat une clarté dans les raisonnements qu'il mettra en place. \newline

L'usage de la calculatrice est autorisé.\newline

\section*{I - Etude d'une fonction}
Dans la suite du sujet, on note $f$ la fonction définie pour tout $x$ par $$f(x) = x^2-x-1$$
\begin{enumerate}
\item Quel est l'ensemble de définition de $f$ ?
\item Calculer les racines de $f$. La racine positive de $f$ est noté $\varphi$ et la négative est noté $\psi$
\item Dresser le tableau de signe de $f$
\item Dresser le tableau de variation de $f$
\item Tracer la courbe représentative de $f$
\item Montrer de deux manières différentes que $$\varphi^2 = \varphi +1$$
\item Montrer que $$\dfrac{1}{\varphi} = \varphi -1$$
\item On note $(u_n)$ la suite définie par $$\left\lbrace \begin{array}{l}
u_0 = 1\\
\forall n \in \N, u_{n+1} = 1 + \dfrac{1}{u_n}
\end{array}
\right.$$
On admet que $(u_n)$ admet une limite $\ell$ lorsque $n$ tend vers l'infini. Que dire de la limite de $(u_{n+1})$ ?
\item Montrer alors que 
$$\ell = \varphi$$
\item Calculer, à l'aide de la calculatrice, $u_{50}$. Que constatez vous ?
\end{enumerate}
\section*{II - Quelques résultats sur les rationnels et les irrationnels}
On rappelle qu'un nombre $x \in \R$ est dit rationnel s'il existe $p \in \Z$ et $q\in \N^*$  tels que $$x = \dfrac{p}{q}$$ L'ensemble des rationnels est noté $\Q$. L'ensemble $\R \backslash \Q$ est l'ensemble des irrationnels.\newline

Tout nombre rationnel s'écrit $$\dfrac{p}{q}$$ avec $p\in \Z$, $q\in \N*$ et $p$ et $q$ premier entre eux c'est à dire pgcd($p$,$q$)=1)
\begin{enumerate}
\item Montrer que le produit de deux rationnels est un rationnel
\item Montrer que la somme de deux rationnels est un rationnel
\item Montrer que l'inverse d'un rationnel non nul est un rationnel
\item En déduire que si $a$ est irrationnel et $b$ est rationnel alors $a\times b$ est forcément un irrationnel
\item Montrer que si $a$ est irrationnel et $b$ est rationnel alors $a+b$ est forcément irrationnel
\item La somme de deux irrationnels est-elle irrationnelle ?
\item Le produit de deux irrationnels est-il irrationnel ?
\item On dit qu'un nombre $n \in \N$ est pair s'il existe $k\in\N$ tel que $n= 2k$. On dit qun nombre $n\in \N$ est impair s'il existe $k\in\N$ tel que $n=2k+1$. Montrer que $$n \text{ impair} \Rightarrow n^2 \text{ impair}$$
Comment s'appelle la raisonnement qui permet de déduire que 
$$\boxed{n^2 \text{ pair} \Rightarrow n \text{ pair}}$$
\item On va montrer que $\sqrt{2}$ est irrationnel. On suppose par l'absurde que $\sqrt{2}$ est irrationnel et donc on suppose l'existence de $p \in \N^*$ et $q \in \N^*$ tel que $$\sqrt{2} = \dfrac{p}{q}$$ avec $p$ et $q$ premiers entre eux
\begin{enumerate}
\item Justifier que $p\in \N^*$
\item Montrer que 
$$p^2 = 2 q^2$$
En déduire que $p$ est pair
\item Montrer alors, en remplaçant $p$ par un certain $2k$ que $q$ est pair
\item Quelle contradiction vient d'arriver ? On en déduit que $\sqrt{2}$ est irrationnel
\end{enumerate}
\item En utilisant une démonstration analogue, on peut montrer que $\sqrt{5}$ est irrationnel. En utilisant les questions 4 et 5, montrer que $\varphi$ est irrationnel
\end{enumerate}
\section*{III - Suites récurrentes linéaires d'ordre 1}
Soit $a$ et $b$ deux réels. On s'interesse aux suites définies par $$\left\lbrace \begin{array}{l}
u_0 = u_0 \quad (\text{donné})\\
\forall n \in \N, u_{n+1} = au_n +b
\end{array}
\right.$$
\begin{enumerate}
\item On s'interesse d'abord au cas $a=1$. Comment appelle-t-on alors ce type de suite. Donner le terme général de $u_n$. Que dire de la limite ?
\item On suppose maintenant $a\neq 1$. On définie la suite $(v_n)$ par $$v_n = u_n - \dfrac{b}{1-a}$$
Montrer que $(v_n)$ est géométrique de raison $a$ et de premier terme que l'on précisera
\item En déduire une expression du terme général de $(v_n)$ puis une expression du terme général de $(u_n)$
\item On suppose ici $0 < a < 1$. Trouver alors la limite de $u_n$ 
\end{enumerate}
\section*{IV - Suites récurrentes linéaires d'ordre 2}
Soit $a$ et $b$ dans $\R$. On dit qu'une suite $u_n$ vérifie $E_{(a,b)}$ si pour tout $n\in \N$, 
$$u_{n+2} = au_{n+1} + bu_n$$
De plus on note 
$$g_{(a,b)} : x\mapsto x^2-ax-b$$ et on note $\Delta_{(a,b)}$ son discriminant
\begin{enumerate}
\item On suppose tout d'abord que $\Delta_{(a,b)} > 0$. On admet alors que les suites $(u_n)$ qui vérifient $E_{(a,b)}$ sont de la forme 
$$u_n = \lambda r_1^n + \mu r_2^n $$
où $\lambda$ et $\mu$ sont des réels quelconque et $r_1$, $r_2$ sont les deux racines de $g_{(a,b)}$\newline


Exprimer alors $\lambda$ et $\mu$ en fonction de $u_0$, $u_1$, $a$ et $b$
\item On suppose maintenant que $\Delta_{(a,b)} = 0$. On admet alors que les suites $(u_n)$ qui vérifient $E_{(a,b)}$ sont de la forme 
$$u_n = (\lambda + \mu n)r_0^n $$
où $\lambda$ et $\mu$ sont des réels quelconque et $r_0$ est l'unique racine de $g_{(a,b)}$ (on suppose $r_0 \neq 0$)\newline


Exprimer alors $\lambda$ et $\mu$ en fonction de $u_0$, $u_1$, $a$ et $b$
\end{enumerate}
\section*{V - Suite de Fibonacci}
On définit la suite de Fibonacci $(\mathcal{F}_n)$ par 
$$\left\lbrace \begin{array}{l}
u_0 = 0 \\
u_1 = 1\\
\forall n \in \N, \mathcal{F}_{n+2} = \mathcal{F}_{n+1} + \mathcal{F}_n
\end{array}
\right.$$
\begin{enumerate}
\item Calculer les 11 premiers termes de la suite de Fibonacci (de $\mathcal{F}_0$ à $\mathcal{F}_{10}$)
\item Exprimer, à l'aide des parties I et III, le terme général de $\mathcal{F}_n$ Cette formule s'appelle la formule de ($\rho$) Binet
\item Montrer que $$\dfrac{\mathcal{F}_{n+1}}{\mathcal{F}_n} \underset{n \rightarrow \infty}{\longrightarrow} \varphi$$
(\emph{Indication : on pourra commencer par remarquer que $|\psi| < 1$})
\item Calculer $\dfrac{\mathcal{F}_{10}}{\mathcal{F}_{9}}$. Que remarquez-vous ?
\end{enumerate}
\end{document}