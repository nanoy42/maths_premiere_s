\documentclass{article}[11pt]
\usepackage[frenchb,english]{babel}
\usepackage[T1]{fontenc}
\usepackage[utf8]{inputenc}
\usepackage{amsmath,amssymb,latexsym}
\usepackage{times}
\usepackage{float}
\usepackage[left=2cm,right=2cm,top=2cm,bottom=2cm]{geometry}
\frenchbsetup{StandardLists=true} % � inclure si on utilise \usepackage[french]{babel}
\usepackage{enumitem}
\usepackage{fancyhdr}
\usepackage{mathrsfs}
\usepackage{graphicx}
%\usepackage[Algorithme]{algorithm}
%\usepackage{algorithmic}
\usepackage{tikz}
\usepackage{tabularx}
\usetikzlibrary{shapes}
\pagestyle{fancy}
\newcommand{\tr}[1]{{\vphantom{#1}}^{\mathit t}{#1}} 
\renewcommand\headrulewidth{1pt}
\fancyhead[L]{Cours 1�re S}
\fancyhead[R]{Yoann Pietri}
\newcounter{theoremecounter}[subsection]
\usepackage{titlesec}
\setcounter{secnumdepth}{3}% enl�ve la num�rotation apr�s les sections
%\renewcommand\thechapter {\Roman{chapter}}

 \setlength{\parindent}{0pt}

\newcommand{\R}{\mathbb{R}}
\newcommand{\N}{\mathbb{N}}
\newcommand{\Q}{\mathbb{Q}}
\newcommand{\Z}{\mathbb{Z}}
\newcommand{\C}{\mathbb{C}}
\newcommand{\K}{\mathbb{K}}
\newcommand{\eqi}{\Leftrightarrow}
\titleformat{\subsubsection}
   {\normalfont\fontsize{11pt}{13pt}\selectfont\bfseries}% apparence commune au titre et au num�ro
   {\thesubsubsection}% apparence du num�ro
   {1em}% espacement num�ro/texte
   {}% apparence du titre

\tikzstyle{theobox} = [draw=black, very thick,
    rectangle, rounded corners, inner sep=10pt, inner ysep=20pt]
\tikzstyle{theotitle} =[fill=white, text=black,rounded corners,draw=black,very thick]


\usepackage{empheq}
\usepackage[c]{esvect}
\newcommand{\covec}[2]{\begin{pmatrix}#1 \\#2 \end{pmatrix}}

\fancyhead[L]{Sujet : Fonction exponentielle}


\begin{document}
\center{Sujet : Etude d'une fonction particulière : la fonction exponentielle}
\flushleft
Si au cours du sujet, le candidat repère ce qui lui semble être une erreur énoncé, il l'indique sur sa copie et continue sa composition. \newline

Il est demandé au candidat une clarté dans les raisonnements qu'il mettra en place. \newline

L'usage de la calculatrice est autorisé.\newline

$$\star$$
\underline{Notations :}\newline
Pour une fonction $f$ quelconque, on note $\mathscr{D}_f$ son ensemble de définition. On note $Im(f) = \{f(x)/x\in \mathscr{D}_f\}$ l'ensemble des images de $f$ (le "/" veut dire "tel que" : c'est l'ensemble des $f(x)$ tel que $x$ soit dans l'ensemble de définition de $f$).\newline

Pour une fonction $f$ on note $f'$ sa dérivée
$$\star$$
\section*{I - Dérivée d'une composée}
\begin{enumerate}
\item R.O.C : Soit $u$ une fonction et $a\in \mathscr{D}_u$, donner la définition de la dérivabilité de $u$ et en $a$ et le cas échéant la formule donnant $u'(a)$
\item On considère deux fonctions $f$ et $g$ telles que $Im(f) \subset \mathscr{D}_f$. On note $h = f(g)$\newline

On va établir le résultat suivant : si $g$ est dérivable sur $I$ et si $f$ est dérivable sur $g(I) = \{g(x) / x\in I\}$, alors $h$ et dérivable sur $I$ et pour tout $x\in I$, $$h'(x) = g'(x)f'(g(x))$$
Soit $a \in I$. On admet que $$\underset{p\rightarrow 0}{\lim}\frac{u(a+p) - u(a)}{p} = \underset{x\rightarrow a}{\lim}\frac{u(x) - u(a)}{x-a}$$Pour simplifier la démonstration (même si ce théorème reste vrai dans le cas général), on suppose $f$ non constante.\newline 

En faisant apparaitre $\frac{1}{g(x) - g(a)} \times (g(x) - g(a))$ dans $$\frac{f(g(x)) - f(g(a))}{x- a}$$ montrer que $$ \underset{x\rightarrow a}{\lim}\frac{f(g(x)) - f(g(a))}{x -a} = g'(a)f'(g(a))$$ En déduire le résultat\newline

On retiendra, pour la suite, le résultat : si $f$ et $g$ sont dérivables sur $\R$ et $h:x\mapsto f(g(x))$, alors, pour tout $x$
$$\boxed{h'(x) = g'(x)\times f'(g(x))}$$
\item Calculer la dérivée de $x\mapsto \sqrt{1+x^2}$ sur $]-1,+\infty[$
\item Soit $f$ une fonction dérivable sur $\R$. Montrer que $$\boxed{(f(-x))' = -f'(-x)}$$(Plus formellement, la dérivée de $x\mapsto f(-x)$ est $x\mapsto -f'(-x)$)
\end{enumerate}
\section*{II - Etude d'une équation}
On s'intéresse à l'équation $$y' = y$$ où $y$ est une fonction définie et dérivable sur $\R$ (l'équation s'écrit aussi $f' = f$)
On s'intéresse plus particulièrement au système 
$$(S) : \left\{ \begin{array}{l}y'=y \\ y(0)=1 \end{array}\right.$$
(On cherche une fonction dérivable sur $\R$ qui vaut $1$ en $0$ et dont la dérivée est elle même)\newline


On va montrer que si une solution de (S) existe, alors celle-ci est unique\newline

Soit $f_0$ dérivable sur $\R$. On suppose $f$ solution de $(S)$. Ainsi $f' = f$ et $f(0) = 1$\newline

On montre tout d'abord que $f_0$ ne s'annule pas sur $\R$. Pour cela, on pose $g$ la fonction $g : x\mapsto f(x)f(-x)$
\begin{enumerate}
\item Montrer que $g$ est constante sur $\R$ (\emph{On pourra utiliser une dérivée et la question I.4)}
\item Donner la valeur de $g$ sur $\R$
\item Conclure que $f$ ne peut s'annuler\newline

On suppose maintenant que $f$ et $h$ sont deux solutions de $(S)$. On considère la fonction $\psi$ sur $\R$ définie par $$\psi(x) = \frac{f(x)}{h(x)}$$
(cette fonction est bien définie car $h$ ne s'annule pas)
\item Montrer que $\psi$ est constante sur $\R$ (\emph{On pourra utiliser une dérivée})
\item Donner la valeur de $\psi$ sur $\R$
\item En déduire que pour tout $x$, $f(x) = h(x)$\newline

\textbf{On admet que le système $(S)$ admet une solution que l'on note $x\mapsto \exp(x)$, appelée fonction exponentielle. Cette fonction est l'unique solution du système $(S)$}\newline


On a $$\forall x \in \R \exp'(x) = \exp(x)$$
$$\exp(0) =1$$

Soit $\alpha \in \R^*$. On considère le système
$$(S_\alpha) : \left\{ \begin{array}{l}y'=y \\ y(0)=\alpha \end{array}\right.$$
\item Monter que si $f$ est solution de $(S_\alpha)$ alors $\frac{1}{\alpha}f$ est solution de $(S)$
\item Montrer que $x\mapsto \alpha\times \exp(x)$ est solution de $(S_\alpha)$. Déduire de la question précédente que c'est la seule solution
\end{enumerate}
\section*{III - Une autre équation}
\begin{enumerate}
\item Soit $\beta \in \R$. Montrer que la dérivée de $x\mapsto \exp(\beta x)$ est $x\mapsto \beta \exp(\beta x)$ (\emph{On utilisera la formule de dérivée d'une composée})
\item Soit $a \in \R$. Déduire de la question précédente que $x\mapsto \exp(ax)$ est solution de l'équation $$(E) \quad y' = a y$$
\item Soit $f$ et $g$ deux solutions de $(E)$. On suppose que $g$ ne s'annule pas et on considère $$h = \frac{f}{g}$$ Montrer que $h$ est constante (\emph{On pourra considérer une dérivée})
\item Déduire que les solutions de $(E)$ sont de la forme $$x \mapsto A \exp(ax)$$ où est $A$ est une constante réelle
\end{enumerate}
\section*{IV - Logarithme népérien}
\textbf{On admet l'existence d'une fonction, appelée logarithme népérien, notée $x\mapsto \ln(x)$ telle que pour tout $x\in \R$, $$\boxed{\ln(\exp(x)) = \exp(\ln(x)) = x}$$}
\begin{enumerate}
\item Montrer que $x \mapsto \ln(\exp(x))$ est dérivable sur $\R$ et donner sa dérivée (en fonction de $\exp$ et $\ln'$. En déduire que 
$$\exp(x) \ln'(\exp(x)) = 1$$
\item En posant $y = \exp(x)$, montrer que $\ln'(y) = \frac{1}{\exp(y)}$\newline

Ainsi la dérivée de $\ln$ est la fonction inverse
\end{enumerate}
\section*{V - Propriétés de l'exponentielle}
\begin{enumerate}
\item On rappelle que $\exp$ ne s'annule pas sur $\R$. Montrer alors que $\exp$ est croissante
\item Soit $a\in\R$. On s'intéresse à la fonction $$\gamma : x\mapsto \exp(a+x)$$Montrer que $\gamma$ est dérivable et calculer sa dérivée
\item Calculer $\gamma(0)$
\item En déduire que $\gamma$ est solution de $(E_{\exp(a)})$
\item Déduire finalement, que pour tout $x$ réel
$$\exp(a+x) = \exp(a)\exp(x)$$
Ainsi pour tout $a,b$ réels $$\boxed{\exp(a+b) = \exp(a)\exp(b)}$$
\item Montrer que $$\exp(a)\exp(-a) =1$$ En déduire que $$\boxed{\exp(-a) = \frac{1}{\exp(a)}}$$ et $$\boxed{\exp(a) = \frac{1}{\exp(-a)}}$$
\item Soit $n\in \N^*$. Montrer que $$\boxed{\exp(nx) = \exp(x)^n}$$ On admet que pour tout $x$ et $y$ réels, $$\exp(xy) = \exp(x)^y$$
\item Montrer les propriétés suivants sur le logarithme népérien : pour tout $a$ et $b$ réels 
$$\ln(ab) = \ln(a) + \ln(b)$$
$$\ln(a^b) = b\ln(a)$$ 
\item Montrer que pour tout $x$ et $y$ réels, $$x^y = \exp(y \ln(x))$$
\end{enumerate}
\section*{VI - Mise sous forme de puissance}
\begin{enumerate}
\item R.O.C. Que valent $(ab)^c$ ? $a^{b+c}$ ? $(a^b)^c$ ? Trouver des analogies avec l'exponentielle
\item On pose $e = \exp(1)$. Montrer que $\ln(e) = 1$
\item Montrer que pour tout $x\in \R$, $$\boxed{e^x = \exp(x)}$$ (\emph{On pourra utiliser la question V.9})
\end{enumerate}
\section*{VII - Approximation de $e$ et représentation graphique}
\begin{enumerate}
\item On considère la suite $u_n$ définie par $$u_n = \left(1+ \frac{1}{n}\right)^n$$
Exprimer $u_n$ sous la forme $e^{\ldots}$ (\emph{utiliser la question V.9})
\item On admet que $$\ln\left(1+\frac{1}{n}\right) \underset{n \rightarrow \infty}{\longrightarrow} 1$$. Montrer alors que $$u_n\underset{n \rightarrow \infty}{\longrightarrow} e$$
\item Calculer $u_{1000}$ à l'aide de la calculatrice. Comparer à la valeur de $\exp(1)$ donner à la calculatrice. Tracer sur la copie les graphes de $x\mapsto\exp(x)$ et $x\mapsto (u_{1000})^x$. La suite $u_n$ converge-t-elle rapidement ou lentement vers $e$
\end{enumerate}
\center
FIN DU SUJET
\flushleft
$$\star \star \star$$
\emph{La fonction exponentielle est surement l'une des fonctions les plus importantes des mathématiques puisqu'on la retrouve un peu partout, tout comme $e$ qui est irrationnel}
\end{document}